
\chapter{Cryptanalyse d'un cryptogramme simple\label{Apx:FBI}}
Pour mettre en pratique l'analyse des fréquences et l'attaque par
mot probable vues au chapitres \ref{sec:AnalyseFrequences} et
\ref{sec:MotProbable}, nous allons
casser un message chiffré\footnote{Disponible sur
\url{http://www.fbi.gov/page2/nov07/code112107.html}}, 
proposé par le FBI sur leur site fin
octobre 2007 (conçus par leurs cryptanalystes « just for fun »)
\\
Le cryptogramme est « PIKODENHFENJIKM! YIH QELB GDISBK NQB
PICB. OI NI AGJ.OIL/PICB.QNT MI WB SKIW, EKC UFBEMB PIKMJCBD E
PEDBBD WJNQ NQB AGJ. »
\\

À première vue, le cryptogramme à tout l'air d'une substitution
monoalphabétique, du fait de la présence d'espaces et de
ponctuation.
Avant tout, essayons de trouver des éléments spéciaux dans le
cryptogramme. On remarque assez vite un E seul vers la fin, ce qui
a beaucoup de chance de correspondre avec la lettre A (« a » étant
le seul mot composé d'une lettre en anglais), ainsi que le «
AGJ.OIL/PICB.QNT », qui ressemble à une URL (sans le préfixe
\emph{www}). Comme ce code est proposé par le site
\url{www.fbi.gov}, on se doute bien que c'est cette même URL.
\\

Faisons ensuite une analyse des fréquences ; les lettres
apparaissant le plus souvent dans le cryptogramme sont I et B (12
fois), N et E (7 fois), et K (6 fois).
Nous savons déjà que I correspond à O en clair, et E correspond à
A. En regardant les fréquences moyenne d'apparition des lettres en
anglais (figure \ref{fig:Frequences}, page
\pageref{fig:Frequences}), la lettre apparaissant le plus est le E, qui devrait
correspondre à B ici, suivie de la lettre T, qui correspond alors
sûrement à N, et enfin (après A et O, que nous connaissons déjà),
viens la lettre N, qui correspond donc sûrement à K ici.
\\

Nous connaissons alors déjà quelques correspondances entre lettres
claires et lettres chiffrées : \\
\begin{center}
  \begin{tabular}{|l|c|c|c|c|c|c|c|c|c|c|}
    \hline
    Lettre claire & A & F & B & I & G & O & V & E & T & N \\
    \hline
    Lettre chiffrée & E & A & G & J & O & I & L & B & N & K \\
    \hline
  \end{tabular}
\end{center}
%\begin{center}
%  \begin{tabular}{|c|c|}
%    \hline
%    Lettre claire & Lettre chiffrée \\
%    A & E \\
%    F & A \\
%    B & G \\
%    I & J \\
%    G & O \\
%    O & I \\
%    V & L \\
%    E & B \\
%    T & N \\
%    N & K \\
%    \hline
%  \end{tabular}
%\end{center}
%
On pourrait encore essayer de trouver plus de lettres via
l'analyse des fréquence des lettres et des bigrammes présent dans
le cryptogramme.
\\

La suite consiste à écrire le message avec les lettres que nous
connaissons déjà, et à essayer de trouver les lettres manquantes.

Ainsi, le premier mot serait « \_ONG\_AT\_\_ATION\_! », et on trouve
assez rapidement que ce mot est « CONGRATULATIONS! », ce qui nous
permet de connaître encore 5 lettres de plus.

Il suffit alors de continuer ainsi de suite jusqu'à avoir
déchiffré le message en entier, dans ce cas ci, le message clair
est « CONGRATULATIONS! YOU HAVE BROKEN THE CODE. GO TO
FBI.GOV/CODE.HTM SO WE KNOW, AND PLEASE CONSIDER A CAREER WITH THE
FBI »
