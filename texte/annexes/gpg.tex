\chapter{La cryptographie en pratique via GPG}
GPG\emph{\url{http://www.gnupg.org/}}, pour 
\emph{GNU Privacy Guard} est une version libre du
standard de cryptographie OpenPGP.

%% TODO histoire ?

Nous allons voir ici comment chiffrer et signer des informations 
grâce à ce
logiciel, via une interface en ligne de commande (il existe
néanmoins de nombreuses interfaces plus intuitives\footnote{voir
\url{http://www.gnupg.org/related_software/frontends.html}}, mais
l'utilisation de la ligne commande permet de mieux comprendre le
fonctionnement du programme, et de mieux le contrôler).
\\

GPG supporte aussi bien le chiffrement symétrique qu'asymétrique
(bien qu'il est très peu utilisé pour le chiffrement symétrique)
au travers de nombreux algorithmes. Pour avoir la liste des
algorithmes supportés, il faut appeler le programme avec comme
argument \texttt{--version} : 

\lstset{language=bash}
\begin{lstlisting}

% gpg --version 
gpg (GnuPG) 1.4.9
Copyright (C) 2008 Free Software Foundation, Inc.
License GPLv3+: GNU GPL version 3 or later
<http://gnu.org/licenses/gpl.html>
This is free software: you are free to change and redistribute it.
There is NO WARRANTY, to the extent permitted by law.

Home: ~/.gnupg
Algorithmes supportés:
Clé publique: RSA, RSA-E, RSA-S, ELG-E, DSA
Chiffrement: 3DES, CAST5, BLOWFISH, AES, AES192, AES256, TWOFISH
Hachage: MD5, SHA1, RIPEMD160, SHA256, SHA384, SHA512, SHA224
Compression: Non-compressé, ZIP, ZLIB, BZIP2

\end{lstlisting}

\section{Chiffrement symétrique}
Chiffrons un fichier contenant du texte, via l'algorithme
blowfish. Pour chiffrer, il faut utiliser le paramètre
\texttt{-c}, et pour préciser un algorithme de chiffrement, on
utilise le paramètre \texttt{--cipher-algo}, suivi du nom de
l'algorithme.\\
Nous avons donc un fichier \texttt{foo.txt}, contenant la phrase
«~Ce message va être chiffré via blowfish~».

\begin{lstlisting}
% echo "Ce message va être chiffré via blowfish" > foo.txt
% gpg -c --cipher-algo blowfish foo.txt
Entrez la phrase de passe: 
Répétez la phrase de passe: 
% cat foo.txt
Ce message va être chiffré via blowfish
% cat foo.txt.gpg
öBï"E`ÉG­¨@\Q3b6Þ}¨ù¯i{Á´J&¯њà÷¤ï;mÈõ%``aW/êÆÉ,*#g&4ÑDì~À¬ãÍG?Fï
                                                                S
\end{lstlisting}

On remarque donc que GPG a créé un fichier \texttt{foo.txt.gpg},
contenant le message chiffré. Avant le chiffrement, GPG nous a
demandé d'entrer une «~phrase de passe~», ce qui correspond à la
clé utilisé pour le chiffrement.
\\

Pour déchiffrer, on appelle alors GPG avec le paramètre
\texttt{-d}, sans devoir préciser l'algorithme utilisé (GPG s'en
chargera) : 

\begin{lstlisting}
% gpg -d foo.txt.gpg 
gpg: données chiffrées avec BLOWFISH
Entrez la phrase de passe: 
gpg: chiffré avec 1 phrase de passe
Ce message va être chiffré via blowfish
gpg: AVERTISSEMENT: l'intégrité du message n'était pas protégée
\end{lstlisting}

GPG affiche donc le contenu du fichier une fois déchiffré (dans le
cas de données, et non d'un texte, on peut utiliser une
redirection de flux (via le mot clé \texttt{>}) pour écrire ces 
données directement dans un fichier.

L'avertissement est normal, il nous préviens qu'une autre personne
aurait pu déchiffrer le message, le modifier et le chiffrer à
nouveau. Pour pallier à cela, GPG met à notre disposition des
mécanismes pour vérifier l'intégrité des données. 

Dans le cas où l'on aurait rentré une mauvaise clé, GPG nous
aurait averti : 
\begin{lstlisting}
% gpg -d foo.txt.gpg
gpg: données chiffrées avec BLOWFISH
Entrez la phrase de passe:
gpg: chiffré avec 1 phrase de passe
gpg: le déchiffrement a échoué: mauvaise clé
\end{lstlisting}
