\chapter{La cryptographie en pratique via GPG}
GPG\footnote{\url{http://www.gnupg.org/}}, pour 
\emph{GNU Privacy Guard} est une version libre du
standard de cryptographie OpenPGP.

Nous allons voir ici comment chiffrer et signer des informations 
grâce à ce
logiciel, via une interface en ligne de commande (il existe
néanmoins de nombreuses interfaces plus intuitives\footnote{voir
\url{http://www.gnupg.org/related_software/frontends.html}}, mais
l'utilisation de la ligne commande permet de mieux comprendre le
fonctionnement du programme, et de mieux le contrôler).
La plupart des fonctions présentées ici sont intégrées dans les
programmes en ayant besoin (les clients mail par exemple), rendant
l'utilisation de GPG via la ligne de commande très peu fréquente.
\\

GPG supporte aussi bien le chiffrement symétrique qu'asymétrique
(bien qu'il est très peu utilisé pour le chiffrement symétrique)
au travers de nombreux algorithmes. Pour avoir la liste des
algorithmes supportés, il faut appeler le programme avec comme
argument \texttt{--version} : 

\lstset{language=bash}
\begin{lstlisting}

% gpg --version 
gpg (GnuPG) 1.4.9
Copyright (C) 2008 Free Software Foundation, Inc.
License GPLv3+: GNU GPL version 3 or later
<http://gnu.org/licenses/gpl.html>
This is free software: you are free to change and redistribute it.
There is NO WARRANTY, to the extent permitted by law.

Home: ~/.gnupg
Algorithmes supportes:
Cle publique: RSA, RSA-E, RSA-S, ELG-E, DSA
Chiffrement: 3DES, CAST5, BLOWFISH, AES, AES192, AES256, TWOFISH
Hachage: MD5, SHA1, RIPEMD160, SHA256, SHA384, SHA512, SHA224
Compression: Non-compresse, ZIP, ZLIB, BZIP2
\end{lstlisting}

\section{Chiffrement symétrique}
Chiffrons un fichier contenant du texte, via l'algorithme
blowfish. Pour chiffrer, il faut utiliser le paramètre
\texttt{-c}, et pour préciser un algorithme de chiffrement, on
utilise le paramètre \texttt{--cipher-algo}, suivi du nom de
l'algorithme. (le paramètre
--armor permet d'avoir du texte «~lisible~» (format ASCII et non
binaire)).
\\
Nous avons donc un fichier \texttt{foo.txt}, contenant la phrase
«~Ce message va être chiffré via blowfish~».

\begin{lstlisting}
% echo "Ce message va etre chiffre via blowfish" > foo.txt
% gpg -c --cipher-algo blowfish --armor foo.txt
Entrez la phrase de passe: 
Repetez la phrase de passe: 
% cat foo.txt.asc
-----BEGIN PGP MESSAGE-----
Version: GnuPG v1.4.9 (GNU/Linux)

jA0EBAMCmeHr5EiOwTdgyUS35guQ8FTtTnHfww0ABklr59UCSHsEzcf0/X+w00Hn
Ui3Ze5FFKSk2NPJw1/KzvowNqt2CcnjpabBZfJf27L3Tccdfvg==
=PjV2
-----END PGP MESSAGE----
\end{lstlisting}

On remarque donc que GPG a créé un fichier \texttt{foo.txt.asc},
contenant le message chiffré (donc illisible). 
Avant le chiffrement, GPG nous a
demandé d'entrer une «~phrase de passe~», ce qui correspond à la
clé utilisé pour le chiffrement.
\\

Pour déchiffrer, on appelle alors GPG avec le paramètre
\texttt{-d}, sans devoir préciser l'algorithme utilisé (GPG s'en
chargera) : 

\begin{lstlisting}
% gpg -d foo.txt.asc 
gpg: donnees chiffrees avec BLOWFISH
Entrez la phrase de passe: 
gpg: chiffre avec 1 phrase de passe
Ce message va etre chiffre via blowfish
gpg: AVERTISSEMENT: l'integrite du message n'etait pas protegee
\end{lstlisting}

GPG affiche donc le contenu du fichier une fois déchiffré (dans le
cas de données, et non d'un texte, on peut utiliser une
redirection de flux (via le mot clé \texttt{>}) pour écrire ces 
données directement dans un fichier.

L'avertissement est normal, il nous prévient qu'une autre personne
aurait pu déchiffrer le message, le modifier et le chiffrer à
nouveau. Pour pallier à cela, GPG met à notre disposition des
mécanismes pour vérifier l'intégrité des données. 

Dans le cas où l'on aurait rentré une mauvaise clé, GPG nous
aurait averti : 
\begin{lstlisting}
% gpg -d foo.txt.asc
gpg: donnees chiffrees avec BLOWFISH
Entrez la phrase de passe:
gpg: chiffre avec 1 phrase de passe
gpg: le dechiffrement a echoue: mauvaise cle
\end{lstlisting}

\section{Chiffrement asymétrique}
GnuPG met à notre disposition un ensemble d'outil utilisant les
clés publiques afin de pouvoir chiffrer et de signer des données.

\subsection{Gérer les clés}
Avant toute chose, il est nécessaire de générer ses clés si ce
n'est pas déjà fait. Ceci se fait via la commande \texttt{gpg
--gen-key}. Nous pouvons alors choisir divers paramètre (le type
de chiffrement, la longueur de la clé, le temps de validité de la
clé, l'identité de la personne, ainsi qu'un mot de passe).
La clé se génère ensuite, et il est conseillé de produire des
évènements systèmes afin d'améliorer le flux de données
aléatoires nécessaire à la formation de la clé.
\\

Nous pouvons alors gérer les clés, le paramètre
\texttt{--list-key} nous listera les clés publiques que l'on
possède, \texttt{--list-secret-keys} fera de même avec les clés
privées. \texttt{--delete-key laclé} ainsi que
\texttt{--delete-secret-key laclé} permettent de supprimer une clé
publique ou privée.
\\

Pour exporter une clé publique dans un fichier (afin de l'envoyer sur un
serveur de clés PGP par exemple), cela se fait via la commande
\texttt{gpg --export --armor laclé > cle.asc} 

L'importation d'une clé se fera alors via la commande \texttt{gpg
--import cle.asc}
\subsection{Chiffrer}
Pour chiffrer un fichier, cela se fait via la commande \texttt{gpg
--recipient laclé --armor --encrypt fichier}. Il faut bien entendu
indiquer la clé publique du récepteur.
Une fois le message reçu, le destinataire pourra le déchiffrer via
la commande \texttt{gpg --decrypt fichier}
\subsection{Signatures}
Pour signer un fichier (afin de prouver que l'on est bien
l'expéditeur du fichier, et que le fichier n'a pas été modifié
depuis la signature), il faut exécuter la commande \texttt{gpg
--default-key lacle --clearsign fichier}.

Pour vérifier l'intégrité du fichier, on appellera GPG ainsi :
\texttt{gpg --verify fichier}
