\chapter{Mise en place d'un XOR en C}
Nous allons voir ici comment mettre en place un chiffrement via le
\emph{ou exclusif}, expliqué dans le chapitre \ref{syst:XOR}, page
\pageref{syst:XOR}.

Le programme que nous allons écrire nous chiffrera un message à partir
d'une clé donnée. Nous devrons donc indiquer deux paramètres aux
programme : le fichier contenant le message ainsi que le fichier
contenant la clé.

Ensuite, le programme parcourera les fichiers simultanéments en mode
binaire, caractère par caractère en effectuant à chaque fois un ou
exclusif entre le caractère du message et celui de la clé. Si nous
arrivons à la fin de la clé avant la fin du message, nous répetons la
clé à partir du début. À chaque caractère chiffré, nous l'affichons
sur la sortie standard.

La mise en place d'un tel programme est assez simple, voici le code
source en C : 
\lstset{language=C}
\lstinputlisting[]{src/xor.c}

L'opération XOR est effectuée à la ligne 25, grâce à l'opérateur \texttt{\^}.

On peut alors facilement utiliser un générateur de nombres
pseudo-aléatoires pour créer une clé, qui sera de la même taille que
le message, et nous aurons donc un masque jetable.

Sur la plupart des systèmes basés sur Unix (Linux, FreeBSD, Solaris,
\dots), de tels générateurs sont présent par défaut ; souvent au
nombre de deux : un plus rapide mais moins aléatoire, et un au
contraire plus aléatoire mais plus lent. Ces générateurs sont
respictivement dans les fichiers spéciaux \texttt{/dev/urandom} et
\texttt{/dev/random}. Utilisons alors la commande \texttt{dd} pour
écrire un fichier de la même taille que le message, contenant des bits
aléatoires venant de \texttt{/dev/random}.

Si le contenu à chiffrer est dans le fichier \texttt{message}, nous le
chiffrerons ainsi (la commande \texttt{du message | awk '{print \$1}'}
nous donne la taille du fichier en octets) : 
\lstset{language=csh}
\begin{lstlisting}
# Creation de la cle
% dd if=/dev/random of=key bs=1 count=`du message | awk '{print $1}'`
# Chiffrement
% ./xor message key > encrypted-message
# Dechiffrement
% ./xor encrypted-message key 
\end{lstlisting}

Pour faciliter le chiffrement et la création de la clé, nous pouvons
utiliser un simple script sh qui fera le travail pour nous : 

\lstset{language=csh}
\lstinputlisting[]{src/use-xor.sh}

Il nous suffira alors d'appeler ce script en lui spécifiant le fichier
contenant le message.\\

Il y a deux inconvénients donc à ce système : le transport de la clé,
qu'on pourrait communiquer via un système de communication possédant
déjà un chiffrement à clé publique. Le second problème est la
génération de la clé ; le générateur \texttt{/dev/random} est fiable
au niveau du contenu aléatoire, mais est extrêmement lent et sa
vitesse dépend de l'activité de la machine.

%TODO: autres générateurs ?
