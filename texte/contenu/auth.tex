\section{L'authentification}
L'authentification est une pratique permettant de vérifier
l'identité de l'utilisateur.

Cette authentification peut se faire sur plusieurs données : 
\begin{itemize}
  \item Une donnée connue par l'utilisateur. Exemple : un mot de
passe ~;
  \item Quelque chose que possède l'utilisateur, un \emph{token}.
Exemple : une carte magnétique ~;
  \item Quelque chose propre à l'utilisateur, une donnée physique.
Exemple : une empreinte digitale ~;
  \item Quelque chose que seul l'utilisateur sait faire. Exemple :
une signature.
\end{itemize}

Avec ces données, il est alors possible d'avoir deux types
d'authentifications : 
\begin{itemize}
  \item L'authentification simple qui consiste simplement à
utiliser un des quatre éléments cités ci-dessus pour authentifier
l'utilisateur. L'élément le plus utilisé est le mot de passe
  \item L'authentification forte, qui 

\end{itemize}
