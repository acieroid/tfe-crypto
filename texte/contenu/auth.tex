\section{L'authentification\label{sec:Authentification}}
L'authentification est une pratique permettant de vérifier
l'identité de l'utilisateur.

Cette authentification peut se faire sur plusieurs données : 
\begin{itemize}
  \item une donnée connue par l'utilisateur. Exemple : un mot de
passe~;
  \item quelque chose que possède l'utilisateur, un \emph{token}.
Exemple : une carte magnétique~;
  \item quelque chose propre à l'utilisateur, une donnée physique.
Exemple : une empreinte digitale~;
  \item quelque chose que seul l'utilisateur sait faire. Exemple :
une signature.
\\
\end{itemize}

Avec ces données, il est alors possible d'avoir deux types
d'authentification : 
\begin{itemize}
  \item l'authentification simple qui consiste simplement à
utiliser un des quatre éléments cités ci-dessus pour authentifier
l'utilisateur. L'élément le plus utilisé est le mot de passe~;
  \item l'authentification forte, qui combine plus d'un élément
des quatre cités ci-dessus.
\\
\end{itemize}

Pour l'authentification forte, on utilise souvent des
\emph{token} (un objet que possède l'utilisateur, servant
uniquement pour l'authentification), 
combinés à l'utilisation traditionnelle de mot de passe, souvent
sous la forme de code PIN, qui permet d'activer le token.
On peut aussi utiliser une carte à puce ou des logiciels présents
sur un PDA, un ordinateur, un téléphone portable, …

Dans le cas où on ne passe pas par un code PIN, ce n'est plus une
authentification forte, mais simple.

\subsection{Les mots de passe}
Les mots de passe sont donc une authentification simple, car ils
ne font appel qu'à un facteur d'authentification : ce que
l'utilisateur connaît. 

Pour sécuriser l'authentification par mot de passe, il faut
sécuriser le mécanisme à plusieurs niveaux :
\begin{itemize}
  \item au niveau de leur forme : les mots de pass doivent 
être résistants à une attaque par force brute ou à une attaque 
par dictionnaire
  \item au niveau de la transmission des mots de passe : en effet,
il est assez simple d'écouter (de \emph{sniffer}) une connexion
réseau et si le mot de passe est transmis en clair, on peut
facilement le retrouver. Le mot de passe doit alors être chiffré
pour la transmission.
  \item au niveau de l'endroit où est stocké le mot de passe car
si un attaquant prend possession d'une machine où sont stockés
des mots de passe, il aura accès à de nombreux mots
de passe directement, alors que, s'ils sont
chiffrés, il ne pourra pas connaître ces mots de passe.
\\
\end{itemize}

Au niveau du chiffrement du mot de passe, il est préférable
d'utiliser une fonction de hachage (fiable), ainsi les seules
attaques possibles sont celles par force brute et par
dictionnaire. La sécurité du mot de passe dépend alors du mot de
passe uniquement, et non de la méthode de chiffrement utilisée.
%%TODO: temps mis par johntheripper sur un mot de passe md5

Il reste néanmoins beaucoup d'autres méthodes pour retrouver un mot de
passe, non basées sur la cryptologie, notamment le hameçonnage
(\emph{phishing}) ainsi que les logiciels malveillants (les
\emph{keylogger} par exemple, qui enregistrent les touches
frappées par l'utilisateur). Il est aussi assez fréquent de voir
que les utilisateurs notent leurs mots de passe sur un papier à
côté de l'ordinateur, utilisent le même mot de passe partout, …

Les mots de passe n'offrent alors plus un niveau de sécurité assez
élevé et l'on privilégie donc l'authentification forte (surtout
pour les services nécessitant une grande sécurité, comme les
banques en ligne, …).

\subsection{Les mots de passe à usage unique (OTP)}
Les mot de passe à usage unique (OTP pour \emph{One Time
Password}) utilisent un secret qui est connu du client et du
serveur. En plus d'un mot de passe, c'est donc un mécanisme
d'authentification forte.
Grâce à ce secret, les deux parties peuvent alors générer
un mot de passe unique, via diverses méthodes : 

\begin{itemize}
  \item l'utilisation d'un compteur : le client ainsi que le
serveur ont possession, en plus du secret, d'un compteur en
commun. À chaque authentification, ce compteur est augmenté, et le
mot de passe produit est donc tout le temps différent~;
  \item l'utilisation du temps : plutôt que de produire le mot de
passe à partir d'un compteur, on se base sur le temps actuel, le
mot de passe sera alors valable pendant une certaine durée
(habituellement courte, le temps de recopier le mot de passe, 
 entre 30 secondes et 1 minute)~;
  \item l'utilisation de \emph{challenge} : le serveur crée un
\emph{challenge} (des données aléatoires), qu'il communique au
client. Le client crée alors le OTP à partir de ce challenge et du
secret connu du serveur et de lui même. Il faut alors éviter de
donner plusieurs fois le même challenge, sinon le mot de passe en
résultant serait le même, et un attaquant pourrait alors
s'identifier en tant qu'utilisateur, s'il a écouté les
communications lors de la première authentification avec le même
challenge. 
\\
\end{itemize}

L'utilisation de challenge (dite asynchrone, car ce n'est pas
immédiat, l'utilisateur doit d'abord rentrer le challenge dans 
son token) pose donc certains problèmes comparée aux autres 
utilisations (dites synchrones). Notons que 
l'utilisation d'un compteur et du temps peut être utilisée de
manière complémentaire.


\subsection{Certificats}
Un certificat numérique est une sorte de carte d'identité,
utilisé pour associer une personne, une machine avec sa clé
publique.
La norme établissant le format de ces certificats est la norme
X.509, elle décrit les éléments qui doivent être présents dans le
certificat (algorithme de signature, nom, dates de validité, …).

Les certificats sont gérés par une autorité de certification (AC),
qui délivre les certificats.
Lors d'une communication entre un utilisateur et un service,
le service envoie son certificat, que l'utilisateur va vérifier
auprès de l'autorité de certification.

%% TODO: plus développer ? PKI ?

\subsection{Biométrie}
Outre les certificats et les \emph{One Time Password}, on peut
utiliser la biométrie pour l'authentification forte, qui consiste
à identifier l'utilisateur via une de ses caractéristiques
physiques (reconnaissance vocale, digitale, faciale et
reconnaissance d'iris).
Cette technique est plus coûteuse à mettre en place et peut poser
des problèmes juridiques (les bases de données contenant les
informations sur les utilisateurs se rapprochent de celles
utilisées par la police, soumises à des règles strictes).
