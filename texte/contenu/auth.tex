\section{L'authentification}
L'authentification est une pratique permettant de vérifier
l'identité de l'utilisateur.

Cette authentification peut se faire sur plusieurs données : 
\begin{itemize}
  \item Une donnée connue par l'utilisateur. Exemple : un mot de
passe ~;
  \item Quelque chose que possède l'utilisateur, un \emph{token}.
Exemple : une carte magnétique ~;
  \item Quelque chose propre à l'utilisateur, une donnée physique.
Exemple : une empreinte digitale ~;
  \item Quelque chose que seul l'utilisateur sait faire. Exemple :
une signature.
\end{itemize}

Avec ces données, il est alors possible d'avoir deux types
d'authentifications : 
\begin{itemize}
  \item l'authentification simple qui consiste simplement à
utiliser un des quatre éléments cités ci-dessus pour authentifier
l'utilisateur. L'élément le plus utilisé est le mot de passe ~;
  \item l'authentification forte, qui combine plus d'un élément
des quatre cités ci-dessus.
\end{itemize}

Pour l'authentification forte, on utilise souvent des
\emph{token} (un objet que possède l'utilisateur), 
combinés à l'utilisation traditionnelle de mot de passe.
Il existe plusieurs alors plusieurs méthodes utilisant des token,
que nous allons voir ci-dessous.

\subsection{Les mots de passe à usage unique (OTP)}
Let token à mot de passe unique (OTP pour \emph{One Time
Password}) utilisent un secret qui est connu du client et du
serveur. Grâce à ce secret, les deux parties peuvent alors générer
un mot de passe unique, via diverses méthodes : 

\begin{itemize}
  \item l'utilisation d'un compteur : le client ainsi que le
serveur ont possession, en plus du secret, d'un compteur, 


\end{itemize}

\subsection{Certificats}
X.509

\subsection{Biométrie}
Outre les certificats et les \emph{One Time Password}, on peut
utiliser la biométrie pour l'authentification forte, qui consiste
à identifier l'utilisateur via une de ses caractéristiques
physiques (reconnaissance vocale, digitale, faciale et
reconnaissance d'iris).
Cette technique est plus coûteuse à mettre en place et peu poser
des problèmes juridiques (les bases de données contenant les
informations sur les utilisateurs se rapprochent de celles
utilisées par la police).
\subsection{Protocoles d'authentifications}

\subsection{Authentification des messages}

\subsection{Échange de clé}
