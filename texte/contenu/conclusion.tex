\chapter{Conclusion}
\thispagestyle{empty}

Comme nous l'avons vu, la cryptologie est une science en
évolution constante, qui a joué un rôle important dans l'histoire.
Il existe de nombreuses techniques de chiffrement ainsi que de
cryptanalyse dont certaines sont plus efficaces que d'autres. Nous avons aussi
vu que la cryptologie fait partie intégrante de notre vie, surtout
dans les systèmes électroniques et informatiques.
\\

La réponse à la problématique de ce travail, qui est «~Peut-on
réellement cacher des informations~», est donc assez mitigée, dans
le sens où il est possible de cacher des informations
facilement, mais difficilement de manière efficace.
Grâce à la publication d'algorithmes de chiffrement (ce qui 
permet de les améliorer, d'ailleurs, un algorithme non
publié n'est pas considéré comme fiable), il est maintenant facile
de chiffrer des informations avec des algorithmes fiables. Néanmoins,
les ordinateurs gagnant en puissance de jour en jour\footnote{D'après la
loi de Moore, la puissance des processeurs double tous les 18
mois, et cette loi se révèle vraie depuis 40 ans.}, le temps mis
pour casser un message chiffré avec un algorithme donné diminue de
jour en jour. %TODO faire un test

Ainsi, on peut donc cacher des informations mais pas
indéfiniment, bien que cette durée puisse être très longue.

D'autre part, les informations peuvent être retrouvées suite à une
erreur humaine (mauvais choix de l'algorithme de chiffrement,
stockage de la clé quelque part, …) 
mais dans les conditions parfaites, où la partie
humaine du processus de chiffrement ne ferait aucune erreur, nous
pouvons affirmer que les données seraient indéchiffrables par une
personne n'ayant pas le droit d'avoir ces informations, pendant
une durée suffisamment longue.

Au niveau des erreurs humaines, il faut veiller à utiliser des
pratiques récentes (algorithmes de chiffrement, logiciels à
jour), ayant
prouvé leur fiabilité. Nous pouvons citer comme exemple les logiciels du
projet OpenBSD\footnote{\url{http://openbsd.org}}, dont le code
source est publié et est en audit de code permanent (plusieurs
personnes relisent le code source à la recherche de failles de
sécurité à corriger).
\\

Nous avons donc essayé d'aborder les points principaux de la
cryptologie dans ce travail mais il est évident que chaque point
pourrait être développé plus en profondeur. Pour plus
d'informations sur la cryptologie, le lecteur intéressé  pourra consulter les livres
cités en références ou les nombreux documents
disponibles sur Internet.

