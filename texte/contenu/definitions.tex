%\newcommand{\glossaire}[2]{\nomenclature{\textbf{#1}}{#2}}
\begin{itemize}

\item {\sffamily\textbf{le chiffrement}} est la démarche effectuée afin de rendre
  le message clair illisible, chiffré. On utilise parfois le terme
  «~cryptage~», qui est un anglicisme, nous éviterons donc de
  l'utiliser~;

\item {\sffamily\textbf{le déchiffrement}} est la démarche inverse du chiffrement, qui retrouve
  le message clair à partir du message chiffré en ayant connaissance
  de la clé, du secret ou de l'algorithme utilisé (contrairement à la
  cryptanalyse). Le mot {\sffamily\textbf{décryptage}} peut aussi
être utilisé~;
\item {\sffamily\textbf{la cryptanalyse}} vise à retrouver le
message clair à partir du message chiffré sans avoir connaissance
de la clé ou du secret~;

\item {\sffamily\textbf{la stéganographie}} est une discipline semblable à la
  cryptographie mais qui consiste à cacher un message (dans une
  image, \dots) et non pas à la rendre initelligible. Nous ne
développerons donc pas ce sujet ici~;

\item {\sffamily\textbf{la clé de chiffrement}} est une donnée (mot, suite d'opération,
  nombre, \dots) utilisée pendant le chiffrement afin de rendre le
  déchiffrement plus difficile sans la connaissance de celle-ci. Il
  existe plusieurs types de clés : certaines qui doivent être gardées
  secrètes (clés privées) et d'autres qui peuvent être diffusées
  (clés publiques)~;

\item {\sffamily\textbf{les données claires}} sont les données dans leur forme initiale, non
  chiffrées. Ces données peuvent être du texte (un message) ou d'autres
  données informatiques (un fichier, \dots)~;

\item {\sffamily\textbf{les données chiffrées}} sont les données qui
sont chiffrées via un certain
  algorithme de chiffrement~;

\item {\sffamily\textbf{un cryptogramme}} est un message chiffré
dans le but d'être déchiffré par des amateurs de cryptographie (on
en retrouve par exemple dans les journaux). Ils sont la plupart du
temps assez simples.
)
\end{itemize}
