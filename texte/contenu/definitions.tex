\newcommand{\glossaire}[2]{\nomenclature{\textbf{#1}}{#2}}

\glossaire{Chiffrement}{la démarche effectuée afin de rendre
  le message clair illisible, chiffré. On utilise parfois le terme
  « cryptage », qui est un anglicisme, nous éviterons donc de
  l'utiliser ;}

\glossaire{Déchiffrement}{le sens inverse du chiffrement, qui retrouve
  le message clair à partir du message chifrfé, en ayant connaissance
  de la clé, du secret ou de l'algorithme utilisé (contrairement à la
  cryptanalyse)}

\glossaire{Décryptage}{synonyme de déchiffrement. Contrairement à «
  cryptage », ce terme est correct et peut être utilisé}

\glossaire{Stéganographie}{une discipline semblable à la
  cryptographie, mais qui consiste à cacher un message (dans une
  image, \dots) et non pas à la rendre initelligible.}

\glossaire{Clé de chiffrement}{une donnée (mot, suite d'opération,
  nombre, \dots) utilisée pendant le chiffrement afin de rendre le
  déchiffrement plus difficile sans la connaissance de celle-ci. Il
  existe plusieurs types de clés, certaines qui doivent être gardées
  secrètes (clés privées), et d'autres qui peuvent être diffusées
  (clés publiques)}

\glossaire{Données claires}{données dans leur forme initiale, non
  chiffrée. Ces données peuvent être du texte (un message) ou d'autres
  données informatiques (un fichier, \dots)}

\glossaire{Données chiffrées}{données chiffrées via un certain
  algorithme de chiffrement}