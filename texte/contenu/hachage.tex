\section{Les fonctions de hachages}
Les fonctions de hachages sont principalement utilisé en cryptographie
pour assurer l'intégrité des données. Une fonction de hachage produit
ce qu'on appelle communément un \emph{hash} ou \emph{somme de
  contrôle}  de taille fixe à partir de
données de taille indéfinie. Ceci peut amener à des \emph{collisions},
c'est à dire que deux données différentes peuvent donner la même somme
de contrôle, elles seront alors considérées comme identiques et cela
peut poser de sérieux problèmes de confidentialitée, de sécurité, ...

Dans ce chapitre, nous allons plus nous focaliser sur les méthodes
utilisées pour calculer la somme à partir des données, nous
regarderons en détail les problèmes de sécurités posés dans le
chapitre \ref{SecuriteHash}, page \pageref{SecuriteHash}.

%TODO

