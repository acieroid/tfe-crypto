\section{Les fonctions de hachages\label{sec:FonctionHachage}}
Les fonctions de hachage sont des fonctions à sens unique, et
produisent à partir d'un texte d'une quelconque longueur, une 
\emph{somme de contrôle} (aussi appelée \emph{empreinte}
ou \emph{hash}) de longueur fixe, qu'on peut voir comme un résumé
du texte (à partir duquel on ne peut pas retrouver le texte
d'origine, ou du moins très difficilement\footnote{Par très
difficile, nous entendons qui n'est pas faisable en pratique, car
cela demanderait des années (voire des centaines d'années, ou
plus) de calculs, même en réunissant la puissance de calculs de
tous les ordinateurs}).

Les fonctions de hachage doivent être résistantes aux collisions,
c'est à dire qu'on ne peut pas trouver facilement deux messages
différents ayant la même empreinte. Bien évidemment, la taille de
l'empreinte étant fixe, il existera toujours des collisions du fait
qu'il y a un nombre limité d'empreintes alors que le nombre de
messages initiaux est illimité.
On peut remarquer que la plupart des fonctions de hachages sont
très sensibles, si l'on change un seul caractère à un message, son
empreinte sera totalement changée.

Nous pouvons vérifier cela avec
la fonction MD5, qui est encore beaucoup utilisée de nos jours,
malgré ses failles de sécurité. Cette fonction est inclue de base
dans la plupart des systèmes Unix, via le programme \texttt{md5sum}.

\lstset{language=bash}
\begin{lstlisting}
% echo "Cryptologie" | md5sum
c457c5d7aedcce9ac99abaa264de9f9e  -
% echo "Cryptologie " | md5sum
c96beec742489f176c84acb07db3ae55  -
% echo "cryptologie" | md5sum
966733eea2d1437b911b4be58ed22c8a  -
\end{lstlisting}

%TODO parler de la construction de l'empreinte ?
