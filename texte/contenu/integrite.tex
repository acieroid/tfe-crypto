
\section{Intégrité et signature}
Comment s'assurer que le prétendu auteur d'un message est
réellement cette personne, et que le message n'a pas été modifié ?
Il faut que l'auteur ait signé son message, et qu'il mette à
disposition une méthode pour vérifier l'intégrité de son message.
\\

Pour vérifier l'intégrité d'un message, on envoie, en plus de ce
message, une empreinte du message. Ainsi, à la réception du
message, on recalcule cette empreinte, et si elle correspond à
l'empreinte envoyée en même temps que le message, le message est
donc authentique. Par contre, si l'attaquant modifie le message,
il peut alors aussi modifier l'empreinte. L'expéditeur chiffre
alors cette empreinte avec sa clé privée, ainsi personne ne peut
changer cette empreinte car personne n'est en possession de sa clé
privée. Le destinataire déchiffrera alors l'empreinte à l'aide de
la clé publique de l'expéditeur, et la comparera avec celle qu'il
aura calculé lui même.

Ainsi, on est à la fois sûr que le message n'a pas été modifiée,
mais en plus on sait qui est l'expéditeur, car uniquement cette
personne a connaissance de sa clé privée. C'est donc le mécanisme
de signature numérique.
\\
%% TODO: parler des PKI

En outre, l'intégrité permet aussi de vérifier que le
téléchargement d'un fichier s'est bien passé. Il faut alors faire
l'empreinte du fichier téléchargé et la comparer avec l'empreinte
qu'a le fichier d'origine (donnée par le miroir qui distribue le
fichier normalement). Si ces deux empreintes correspondent, les
deux fichiers sont donc identiques. À titre d'exemple, le protocole
de P2P BitTorrent vérifie l'empreinte SHA-1 de chaque bout de donnée
du fichier 
(le fichier a télécharger est divisé plusieurs petites parties, 
appelées \emph{chunks}).  

Il est alors important d'utiliser une fonction qui ne comporte pas
de problème de sécurité, sinon un fichier avec du code malicieux
pourrait avoir la même empreinte que le fichier que l'on souhaite
télécharger. Néanmoins, l'algorithme MD5 reste fortement utilisés
malgré ses faiblesses.

