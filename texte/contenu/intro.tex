\chapter{Introduction}
\thispagestyle{empty}

\section{Présentation du travail}
La cryptographie fait aujourd'hui partie de la vie de tous les jours,
on l'utilise sans le savoir, principalement en naviguant sur internet
(via les connexions sécurisées, les envois de mails chiffrés\dots).

Mais qu'est-ce que la cryptographie ? Comment cela
fonctionne-t-il ? Son utilisation se limite-t-elle à l'informatique ?
Depuis quand existe la cryptographie ?

C'est à ces questions que nous allons essayer de répondre durant ce
travail, et plus principalement à celle-ci :

\begin{quote}
\emph{«~Peut-on réellement cacher des informations ?~»}
\end{quote}

Nous essayerons donc de savoir si l'on peut, et comment on peut 
envoyer sans risque une information, un message, des données\dots à
quelqu'un, sans qu'une personne extérieure puisse le lire.

Pour répondre à ces questions, nous allons tout d'abord parler de
l'utilisation  de la cryptographie
à travers l'histoire, pour présenter ensuite les
différentes techniques existantes. Nous parlerons
aussi du moyen de « contrer » la cryptographie, avec la
cryptanalyse. Nous verrons enfin ce qu'il en est de la
cryptographie de nos jours. 

% La cryptographie est, comme le dit si bien Lacroix dans le titre de
% son livre ``La cryptographie, ou l'art d'écrire en chiffres''
% \cite{ArtDecrireEnChiffres},
% un ensemble de méthodes mathématiques ou autres, par lesquelles on va ``crypter'' un message, de façon à le rendre illisible si on n'a pas connaissance de la méthode utilisée, ou de l'éventuelle clée utilisée. \\
% Ce travail aura pour but de savoir si la cryptographie est un moyen fiable pour cacher des données, via l'analyse de certains algorithmes utilisés de nos jours ou dans le passé, et plus précisément leur cryptanalyse (ce terme sera définit à la prochaine section du document) \\
% Nous analyserons donc, d'abord les méthodes de cryptographie utilisée dans l'histoire, suivant un ordre chronologique, ensuite nous passerons aux méthodes actuelles, où nous expliquerons entre autres l'utilité de la cryptographie de nos jours. \\

%En annexe, on pourra voir quelques implémentations d'algorithmes de chiffrement dans certains langages de programmation (Haskell, C, Common Lisp, Python)\footnote{Pour plus d'infos sur ces langages, visitez respectivement \url{http://www.haskell.org}, \url{http://fr.wikipedia.org/wiki/Langage_C}, \url{http://cliki.net} et \url{http://www.python.org}.} \\

\section{Mes motivations pour ce travail}
J'ai choisi de faire un travail de fin d'études sur la cryptographie
car j'ai toujours été intéressé par les mathématiques et surtout par
l'informatique (plus le fonctionnement de l'ordinateur et sa
programmation, que l'aspect « jeux et divertissements » de
l'informatique, ainsi que la sécurité informatique)~;
la cryptographie fait partie intégrante des
systèmes informatiques d'aujourd'hui et est basée sur des principes
mathématiques. Je compte d'ailleurs faire des études polytechniques
l'année prochaine. \\ L'idée de ce travail m'est venue petit à
petit. Après la lecture d'un article sur la cryptographie sur
internet, j'ai commencé à m'y intéresser un peu, mais sans plus.
 Ce travail me permet donc d'approfondir le sujet.

\section{À propos de ce document}
Pour des raisons personnelles et dans un esprit de liberté de
l'information, ce document est placé sous licence Creative Commons
BY-SA\footnote{\url{http://creativecommons.org/licenses/by-sa/2.0/fr/}}, ce qui
signifie que vous pouvez le modifier et le redistribuer librement à
condition de préciser le nom de l'auteur, et de garder cette même
licence. Toujours pour les mêmes raisons, ce document a été préparé à
l'aide d'outils entièrement libres.
%de l'éditeur de texte \texttt{GNU
%  Emacs} et du logiciel de composition typographique \LaTeX, et des
%logiciels \texttt{xfig}, \texttt{dia} et \texttt{inkscape} pour les
%dessins, sur un système d'exploitation basé sur \texttt{Linux}. Tous
%ces logiciels sont entièrement libres. De plus, les langages utilisés
%en annexe possèdent aussi chacun au moins une implémentation
%libre.
Pour plus d'informations sur les logiciels libres et la culture
libre, référez-vous à \url{http://www.gnu.org/philosophy/philosophy.fr.html}. \\
Ce document, ses sources (les fichiers \TeX, les graphiques, et les
codes sources des annexes), ainsi que des informations supplémentaires
sont disponibles sur internet via l'adresse
\url{http://acieroid.tuxfamily.org/crypto}. \\


\section{Présentation de la cryptologie}
Étymologiquement, la cryptologie est la «~science du secret~»~;
Elle consiste à cacher une
information, un message ou de quelconques données via la
\emph{cryptographie}. Nous étudierons son histoire
dans le chapitre \ref{chap:Histoire}.

La cryptographie comprend des méthodes très variées (des
\emph{systèmes de chiffrement} pour rendre les messages chiffrés),
certaines méthodes étant plus efficaces que les autres.
Nous examinerons ces méthodes dans le chapitre
\ref{chap:Techniques}.

La cryptologie comprend aussi la \emph{cryptanalyse}, qui consiste à
retrouver le message d'origine à partir d'un message chiffré sans en
connaître la clé ou le secret utilisé pour chiffrer le message.
Nous verrons cet aspect dans le chapitre \ref{chap:Cryptanalyse}.

Au cours de ce travail, nous utiliserons de nombreux termes se référant à
la cryptologie. \\
Définissons-en quelques uns : 

%\newcommand{\glossaire}[2]{\nomenclature{\textbf{#1}}{#2}}
\begin{itemize}

\item {\sffamily\textbf{le chiffrement}} est la démarche effectuée afin de rendre
  le message clair illisible, chiffré. On utilise parfois le terme
  «~cryptage~», qui est un anglicisme~; nous éviterons donc de
  l'utiliser~;

\item {\sffamily\textbf{le déchiffrement}} est la démarche inverse du chiffrement, qui retrouve
  le message clair à partir du message chiffré en ayant connaissance
  de la clé, du secret ou de l'algorithme utilisé (contrairement à la
  cryptanalyse). Le mot {\sffamily\textbf{décryptage}} peut aussi
être utilisé~;
\item {\sffamily\textbf{la cryptanalyse}} vise à retrouver le
message clair à partir du message chiffré sans avoir connaissance
de la clé ou du secret~;

\item {\sffamily\textbf{la stéganographie}} est une discipline semblable à la
  cryptographie mais qui consiste à cacher un message (dans une
  image\dots) et non pas à la rendre inintelligible. Nous ne
développerons donc pas ce sujet ici~;

\item {\sffamily\textbf{la clé de chiffrement}} est une donnée (mot, suite d'opération,
  nombre\dots) utilisée pendant le chiffrement afin de rendre le
  déchiffrement plus difficile sans la connaissance de celle-ci. Il
  existe plusieurs types de clés : certaines qui doivent être gardées
  secrètes (clés privées) et d'autres qui peuvent être diffusées
  (clés publiques)~;

\item {\sffamily\textbf{les données claires}} sont les données dans leur forme initiale, non
  chiffrées. Ces données peuvent être du texte (un message) ou d'autres
  données informatiques (un fichier,\dots)~;

\item {\sffamily\textbf{les données chiffrées}} sont les données qui
sont chiffrées via un certain
  algorithme de chiffrement~;

\item {\sffamily\textbf{un cryptogramme}} est un message chiffré
dans le but d'être déchiffré par des amateurs de cryptographie (on
en retrouve par exemple dans les journaux). Ils sont la plupart du
temps assez simples.
\end{itemize}

