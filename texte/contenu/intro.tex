\chapter{Introduction}

\section{Présentation du travail}
La cryptographie fait aujourd'hui partie de la vie de tous les jours,
on l'utilise sans le savoir, principalement en naviguant sur internet
(via les connexions sécurisées, les envois de mails chiffrés, \dots).

Mais qu'est ce que la cryptographie ? Comment est-ce que cela
fonctionne ? Son utilisation se limite-t-elle juste à l'informatique ?
Depuis quand existe la cryptographie ?

Ce sont à ces questions que nous allons essayer de répondre durant ce
travail, et plus principalement à cette question :

\begin{quote}
\emph{« Peut-on réellement cacher des informations »}
\end{quote}

Nous essayerons donc de savoir si l'on peut, et comment peut-on
envoyer sans risque une information, un message, des données, \dots à
quelqu'un, sans qu'une personne extérieure puisse le lire.

Pour répondre à ces questions, nous allons tout d'abord parler de son
utilisation à travers l'histoire, pour ensuite présenter les
différentes techniques de cryptographie existantes. Nous parlerons
aussi du moyen de « contrer » la cryptographie, avec la
cryptanalyse. Nous verrons enfin pour finir ce qu'il en est de la
cryptographie de nos jours. Tout au cours du travail, nous
expliquerons certaines méthodes de chiffrement célèbres.

% La cryptographie est, comme le dit si bien Lacroix dans le titre de
% son livre ``La cryptographie, ou l'art d'écrire en chiffres''
% \cite{ArtDecrireEnChiffres},
% un ensemble de méthodes mathématiques ou autres, par lesquelles on va ``crypter'' un message, de façon à le rendre illisible si on n'a pas connaissance de la méthode utilisée, ou de l'éventuelle clée utilisée. \\
% Ce travail aura pour but de savoir si la cryptographie est un moyen fiable pour cacher des données, via l'analyse de certains algorithmes utilisés de nos jours ou dans le passé, et plus précisément leur cryptanalyse (ce terme sera définit à la prochaine section du document) \\
% Nous analyserons donc, d'abord les méthodes de cryptographie utilisée dans l'histoire, suivant un ordre chronologique, ensuite nous passerons aux méthodes actuelles, où nous expliquerons entre autres l'utilité de la cryptographie de nos jours. \\

%En annexe, on pourra voir quelques implémentations d'algorithmes de chiffrement dans certains langages de programmation (Haskell, C, Common Lisp, Python)\footnote{Pour plus d'infos sur ces langages, visitez respectivement \url{http://www.haskell.org}, \url{http://fr.wikipedia.org/wiki/Langage_C}, \url{http://cliki.net} et \url{http://www.python.org}.} \\

\section{Mes motivations pour ce travail}
J'ai choisis de faire un travail de fin d'études sur la cryptographie
car j'ai toujours été intéressé par les mathématiques et surtout par
l'informatique (plus au fonctionnement de l'ordinateur, à sa
programmation qu'à l'aspect « jeux et divertissements » de
l'informatique), et la cryptographie fait partie intégrante des
systèmes informatiques d'aujourd'hui et est basée sur des principes
mathématiques. D'ailleurs, je compte faire des études polytechniques
l'année prochaine. \\ L'idée de ce travail m'est venue petit à
petit. Après la lecture d'un article sur la cryptographie sur
internet, j'ai commencé à m'y intéresser un peu, sans trop
approfondir, et ce travail me permet donc d'approfondir le sujet.

\section{À propos de ce document}
Pour des raisons personnelles, et dans un esprit de liberté de
l'information, ce document est placé sous licence GFDL, ce qui
signifie que vous pouvez le modifier et le redistribuer librement à
condition de préciser le nom de l'auteur, et de garder cette même
licence. Toujours pour les mêmes raisons, ce document a été préparé à
l'aide d'outils entièrement libres.
%de l'éditeur de texte \texttt{GNU
%  Emacs} et du logiciel de composition typographique \LaTeX, et des
%logiciels \texttt{xfig}, \texttt{dia} et \texttt{inkscape} pour les
%dessins, sur un système d'exploitation basé sur \texttt{Linux}. Tous
%ces logiciels sont entièrement libres. De plus, les langages utilisés
%en annexe possèdent aussi chacun au moins une implémentation
%libre.
Pour plus d'informations sur les logiciels libres et la culture
libre, référez vous à \url{http://www.gnu.org/philosophy/philosophy.fr.html}. \\
Ce document, ses sources (les fichiers \TeX~, les graphiques, et les
codes sources des annexes), ainsi des informations supplémentaires
sont disponibles sur internet via l'adresse
\url{http://www.acieroid.tuxfamily.org/tfe/}. \\


\section{Présentation de la cryptologie}
La cryptologie est la « science du secret », et consiste a cacher une
information, un message, ou de quelconques données via la
\emph{cryptographie}. 

La cryptographie comprend des méthodes très variées (des
\emph{systèmes de chiffrement} pour rendre les messages chiffré,
certaines méthodes plus ou moins efficaces que les autres.

La crypptologie comprends aussi la \emph{cryptanalyse}, qui consiste à
retrouver le message d'origine à partir d'un message chiffré, sans en
connaître la clé ou le secret utilisé pour chiffrer le message.

Au cours du travail, nous utiliserons de nombreux termes se référant à
la cryptologie, ceux-ci sont définis dans le glossaire en fin de travail.