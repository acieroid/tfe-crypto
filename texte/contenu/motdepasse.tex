\section{Les mots de passe}
Les mots de passes sont omniprésents sur les systèmes
informatiques, ils doivent donc être introuvables par une
quelconque personne.

Il faut donc qu'ils soient sécurisés à plusieurs niveaux : 
\begin{itemize}
  \item Au niveau de leur forme d'abord, en effet, ils doivent
être résistant à une attaque par force brute ou à une attaque 
par dictionnaire
  \item Au niveau de la transmission des mots de passe, en effet
il est assez simple d'écouter (de \emph{sniffer}) une connexion
réseau, et si le mot de passe est transmit en clair, on peut
facilement le retrouver. Le mot de passe doit alors être chiffré
pour la transmission.
  \item Au niveau de l'endroit où est stocké le mot de passe, car
si un attaquant prends possession d'une machine où sont stockés
des mots de passe, il aura accès à des centaines, milliers de mots
de passes directement, alors que si les mots de passes sont
chiffrés il ne pourra pas connaître ces mots de passe.
\end{itemize}

Bien que l'on entend souvent que le problème est situé «~ entre la
chaise et le clavier ~», de nombreux systèmes informatiques ou
sites internet ne respectent pas ces mesures. Néanmoins, il reste
quand même un grand effort à faire de la part de l'utilisateur :
choisir un mot de passe aléatoire, le retenir sans le noter, …

Au niveau du chiffrement du mot de passe, il est préférable
d'utiliser une fonction de hachage (fiable), ainsi les seules
attaques possibles sont celles par force brute et par
dictionnaire. La sécurité du mot de passe dépend alors du mot de
passe uniquement, et non de la méthode de chiffrement utilisée.

Il reste néanmoins beaucoup d'autres méthodes pour retrouver un mot de
passe, non basées sur la cryptologie, notamment le hameçonnage
(\emph{phishing}) ainsi que les logiciels malveillant (les
\emph{keylogger} par exemple, qui enregistrent les touches
frappées par l'utilisateur).
Les mots de passe n'offrent alors plus un niveau de sécurité assez
élevé, et l'on privilégie donc l'authentification forte.

