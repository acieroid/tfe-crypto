\section{Les niveaux d'attaques}
Un attaquant peu avoir accès à plusieurs informations avant de
commencer la cryptanalyse d'un message chiffré.

Tout d'abord, il faut considérer que l'attaquant connaît en détail
la méthode de chiffrement, la seule partie qu'il ne connaît pas
est la clé (c'est le principe de
Kerckhoffs\footref{sec:PrincipeKerchoffs}). Son but est donc de
retrouver cette clé, et il pourra ainsi déchiffrer le message.

L'attaquant peu aussi avoir d'autres informations que le texte
chiffré et la connaissance de l'algorithme, il existe alors trois
différents niveau d'attaques.

\begin{itemize}
  \renewcommand{\makelabel}[1]{\sffamily\textbf{#1}}
  \item[L'attaque à texte chiffré seul] :
    l'attaquant dispose d'un ou de plusieurs textes chiffrés, son but
    est de retrouver les messages clairs, et idéalement (mais pas
    nécessairement) de retrouver la clé utilisée pour le chiffrement,
    afin de pouvoir déchiffrer d'autres messages ayant été chiffrés
    avec cette clé.

  \item[L'attaque à texte clair connu] :
    ici, l'attaquant possède en plus des messages chiffrés, les
    messages initiaux. Il doit alors retrouver la clé utilisée pour 
    chiffrer les messages ou un algorithme pour déchiffrer n'importe
    quel autre message chiffré avec cette clé. 

  \item[L'attaque à texte clair choisi] :
    cette attaque est presque identique que celle à texte clair connu,
    sauf que l'attaquant peut choisir quel texte sera chiffré. Cette
    attaque est donc plus facile, car il peut choisir des messages qui
    pourraient lui apporter plus d'informations sur la clé.

    Ce type d'attaque se divise en deux attaques : une où les textes
    clairs sont tous choisis avant le chiffrement, et une où on
    chiffre des textes en fonction des résultats des chiffrements
    précédents

\end{itemize}
