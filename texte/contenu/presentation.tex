\chapter{Présentation de la cryptologie}
\section{Qu'est ce que la cryptologie}
La cryptologie est la « science du secret », et consiste a cacher une
information, un message, ou de quelconques données via la
\emph{cryptographie}. 

Elle comprends aussi la \emph{cryptanalyse}, qui consiste à retrouver
le message d'origine à partir d'un message chiffré, sans en connaître
la clé ou le secret utilisé pour chiffrer le message.
\section{Vocabulaire\label{s:Definitions}}
Avant toute chose, il est nécéssaire de définir quelques termes pour
les clarifier dans l'esprit de chacun, et pour éviter de les confondre
:
\begin{itemize}
\renewcommand{\makelabel}[1]{\sffamily\textbf{#1}}
\item[Le chiffrement] se réfère à la démarche effectuée afin de rendre
  le message clair illisible, chiffré. On utilise parfois le terme
  « cryptage », qui est un anglicisme, nous éviterons donc de
  l'utiliser ;

\item[Le déchiffrement, ou décryptage] est le sens inverse du
  chiffrement, qui retrouve le message clair à partir du message
  chiffré, en ayant connaissance de la clé, du secret ou de
  l'algorithme utilisé (contrairement à la cryptanalyse). Le terme
  « décryptage » est un terme français correct, contrairement au
  « cryptage » ;

\item[La stéganographie] est une discipline semblable à la
  cryptographie, mais qui consiste à « cacher » un message (dans une
  image, ...) et non pas à le rendre inintelligible. Ce sujet étant
  tout aussi vaste que la cryptographie, nous ne l'étudierons pas dans
  ce travail ;

\item[La clé de chiffrement] est une donnée (mot, suite d'opération,
  nombre, ...) utilisée pendant le chiffrement afin de rendre le
  déchiffrement plus difficile sans la connaissance de celle ci.
  Nous verrons qu'il existe deux types de clés : \emph{symétrique}
  (\emph{privées}) et \emph{asymétrique} (\emph{publiques}), et que
  l'utilisation de clé est crucial dans la cryptographie actuelle.
\end{itemize}

\section{Règles typographiques}
Tout au long de ce travail, nous allons utiliser certaines notations : 
\begin{itemize}
\item Une \textbf{clé de chiffrement} sera notée $k$ (pour «  Key  »,
  «  Clé  » en anglais), indicé si nous utilisons plusieurs clés ;
  \item Un \textbf{message en clair} sera noté $M$ (pour «  Message  ») ;
  \item Un \textbf{message chiffré} sera noté $C$ (pour «  Chiffré  ») ;
  \item Une \textbf{fonction de chiffrement} $e$ (pour « Encryption
    », «  Chiffrage  » en anglais) ou $e_k$, selon qu'elle utilise
    une clé pour le chiffrement ;
  \item Une \textbf{fonction de déchiffrement} $d$ (pour
    « Decryption », « Déchiffrement  » en anglais) ou $d_k$,
    similairement à la fonction de chiffrement.
\end{itemize}
L'opération de chiffrement se notera alors comme suit pour un
algorithme de chiffrement avec une clé.
\begin{center}
  \begin{math}
    e_k(M) = C
  \end{math}
\end{center}
Et l'opération de déchiffrement se notera ainsi :
\begin{center}
  \begin{math}
    d_k(C) = M
  \end{math}
\end{center}
Nous utiliserons aussi la notation : 
\crypt{Cryptographie}{Pelcgbtencuvr}
pour indiquer que le mot « Cryptographie » sera chiffré en
 « Pelcgbtencuvr ». 
  
