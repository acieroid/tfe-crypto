\chapter{Les chiffres de substitution}
Cette méthode consiste à simplement remplacer une lettre, un ensemble
de lettres (un \emph{polygramme}), un mot, ...  par une autre lettre,
polygramme, ou bien par un nombre ou un signe.

On peut distinguer plusieurs sortes de substitutions : 
\begin{itemize}
  \item Les substitutions monoalphabétiques ou simples~; 
  \item Les substitutions polyalphabétiques~; 
  \item Les substitutions polygrammiques~;
  \item Les substitutions tomogrammiques~.
\end{itemize}

\section{Les substitutions monoalphabétiques}
Ce type de substitution consiste simplement à remplacer chaque lettre
de l'alphabet par une autre lettre.

\subsection{Par simple décalage}
On peut par exemple, comme le fait le chiffre de
César\footref{m:ChiffreCesar}, décaler les lettres de l'alphabet d'une
clé $k$.

 Si nous considérons l'alphabet comme un ensemble cyclique où 
chaque lettre correspondrait à un chiffre (une fois arrivé à $Z$ ($= 25$), on 
repart de $A$ ($= 0$) $\rightarrow 25 + 1 = 0$), on peut facilement définir 
ce système de chiffrement : 
\\
\definition{$e_k(c) = (c + k)~ mod~ 26$\\
  où $c$ est le caractère à chiffrer.}

Nous utilisons l'opérateur \emph{modulo}, qui nous donne le reste de
la division entière de deux entiers ($5~ mod~ 2 = 1$), ce qui nous
permet de rester dans un ensemble cyclique. \\
\\

Une application du chiffre de César est, par exemple, le \emph{ROT13}
(de ``rotate alphabet 13 places'' \cite{ROT13HomePage}).
C'est donc simplement un chiffre de César pour lequel la clé $k$ a une
valeur de $13$.

L'interêt du ROT13 réside dans le fait qu'il suffit d'appliquer deux
fois la méthode de chiffrement pour retrouver le texte en clair
($e_{13}(e_{13}(M)) = M$), ainsi, il n'y a qu'une et une seule fonction utilisée
pour le chiffrement et le déchiffrement (donc $e_{13}(x) = d_{13}(x)$).

De ce fait, ROT13 n'est pas du tout pratique pour cacher des
informations, mais est parfois utilisé sur internet, dans les forums,
afin d'empêcher la lecture involontaire d'un texte (qui pourrait
dévoiler une information sur un film par exemple, et gâcherait le
plaisir au gens n'ayant pas vu le film, c'est ce qu'on appelle un
\emph{spoiler})

\subsection{Par remplacement}
Outre le fait de décaler les lettres, on peut aussi remplacer chaque
lettre de l'alphabet par une autre lettre. Pour chiffrer, on peut alors
s'aider d'une table de substitution, comme sur la figure
\ref{fig:substitutionsimple}.

 \begin{figure}[h]
    \begin{center}
    \begin{tabular}{|c|c|c|c|c|c|c|c|c|c|c|c|c|c|c|c|c|c|c|c|c|c|c|c|c|c|}
      \hline
      A & B & C & D & E & F & G & H & I & J & K & L & M & N & O & P & Q & R & S & T
      & U & V & W & X & Y & Z \\
      \hline
      A & Z & E & R & T & Y & U & I & O & P & Q & S & D & F & G & H & J & K & L & M
      & W & X & C & V & B & N \\
      \hline
    \end{tabular}
  \end{center}
  \caption{Exemple simple de table de substitution}
  \label{fig:substitutionsimple}
\end{figure}

Dans cette table de substitution, on peut placer les lettre
aléatoirement, selon une règle, ou en utilisant une clé (qu'on notera
en début de table, et nous recopierons les caractères non utilisés par
après, voir la figure \ref{fig:substitutioncle}).

\begin{figure}[h]
  \begin{center}
    \begin{tabular}{|c|c|c|c|c|c|c|c|c|c|c|c|c|c|c|c|c|c|c|c|c|c|c|c|c|c|}
      \hline
      A & B & C & D & E & F & G & H & I & J & K & L & M & N & O & P & Q & R & S & T
      & U & V & W & X & Y & Z \\
      \hline
      C & R & Y & P & T & O & A & B & D & E & F & G & H & I & J & K & L & M & N  & Q 
      & S & U & V & W & X & Z \\
      \hline
    \end{tabular}
  \end{center}
  \caption{Exemple de table de substitution avec comme clé ``crypto''}
  \label{fig:substitutioncle}
\end{figure}




% \subsection{Les substitutions alphabétiques}
% Les substitution alphabétiques consistent simplement à remplacer une lettre par une autre, on retrouve deux types de substitutions alphabétiques : 
% \subsubsection{Les substitution mono-alphabétique}
% On remplace chaque lettre de l'alphabet par une autre lettre, ou une séquence de chiffre (comme le carré de Polybe, voir \ref{CarrePolybe}, page \pageref{CarrePolybe}), qui restera toujours la même tout au long du cryptage. On peut utiliser une table de substitution pour faciliter le chiffrement. La première ligne correspond au caractère en clair, à chiffrer, et la seconde ligne au caractère chiffré.

% \begin{figure}[h]
%     \begin{center}
%     \begin{tabular}{|c|c|c|c|c|c|c|c|c|c|c|c|c|c|c|c|c|c|c|c|c|c|c|c|c|c|}
%       \hline
%       A & B & C & D & E & F & G & H & I & J & K & L & M & N & O & P & Q & R & S & T
%       & U & V & W & X & Y & Z \\
%       \hline
%       A & Z & E & R & T & Y & U & I & O & P & Q & S & D & F & G & H & J & K & L & M
%       & W & X & C & V & B & N \\
%       \hline
%     \end{tabular}
%   \end{center}
%   \caption{Exemple simple de table de substitution}
%   \label{TableSubstitutionSimple}
% \end{figure}
% De cette façon, suivant la table présente dans la figure \ref{TableSubstitutionSimple}, si on crypte ``bonjour'' : 
% \crypt{bonjour}{zgfpgwk}

% On peut décaler simplement des lettres de l'alphabet, comme c'est le cas pour le chiffre de César (voir \ref{ChiffreCesar}, page \pageref{ChiffreCesar}),
%  on peut utiliser une clé (voir figure \ref{TableSubstitutionCle}), où remplacer simplement les lettres par d'autres lettres arbitrairement (voir figure \ref{TableSubstitutionSimple}).
% \begin{figure}[h]
%   \begin{center}
%     \begin{tabular}{|c|c|c|c|c|c|c|c|c|c|c|c|c|c|c|c|c|c|c|c|c|c|c|c|c|c|}
%       \hline
%       A & B & C & D & E & F & G & H & I & J & K & L & M & N & O & P & Q & R & S & T
%       & U & V & W & X & Y & Z \\
%       \hline
%       C & R & Y & P & T & O & A & B & D & E & F & G & H & I & J & K & L & M & N  & Q 
%       & S & U & V & W & X & Z \\
%       \hline
%     \end{tabular}
%   \end{center}
%   \caption{Exemple de table de substitution avec comme clé ``crypto''}
%   \label{TableSubstitutionCle}
% \end{figure}
% Nous verrons dans le chapitre sur la cryptanalyse (page \pageref{Cryptanalyse})qu'il est facile de casser ces substitution via l'analyse des fréquences
