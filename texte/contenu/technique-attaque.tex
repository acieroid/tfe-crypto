\section{Les différentes attaques}
Une fois en possession des diverses informations, il existe un
très grand nombre d'attaques différentes pour retrouver le message
clair ou la clé utilisée pendant le chiffrement. Voyons les
attaques les plus utilisées.

\subsection{L'analyse des fréquences\label{sec:AnalyseFrequences}}
Exposée pour la première fois par Al-Kindi au IX\ieme~ siècle,
l'analyse fréquentielle permet de déchiffrer des chiffrements
simples. Dans le cas d'une substitution, les
caractères sont représentés par autre chose, mais leur fréquences
d'apparition ne change pas, on peut donc alors facilement
retrouver le message initial, connaissant les fréquences
normale d'apparition des caractères dans la langue du message.

L'analyse des fréquence ne se limite pas aux lettres
individuelles, l'analyse des bigrammes peut aussi être très
pratique.
\\

On peut via cette méthode aussi trouver la langue dans laquelle le
message est écrit, car chaque langue possède une fréquence
d'apparitions des lettres unique (voir la figure \ref{fig:Frequences}).

Bien évidemment, la fréquence d'apparition des lettres peut varier
au sein d'une même langue selon certains critères (par exemple un
message d'origine militaire risque de contenir beaucoup
d'abréviations). Une analyse des fréquence aura donc des résultats
plus probables sur des messages d'une certaine longueur.

\begin{figure}[h]
  \centering
    \includegraphics[width=0.9\textwidth]{plot/Frequences.png}
    \caption{Fréquences d'apparition des lettres en français et en
anglais}
  \label{fig:Frequences}
\end{figure}


\subsection{L'indice de coïncidence}
Inventé en 1920 par William Friedman, l'indice de coïncidence
permet de découvrir à quel type de chiffrement on a affaire 
, et dans le cas d'une
substitution polyalphabétique, il permet de trouver la taille de
la clé.

Cet indice représente en fait la probabilité que deux lettres
choisies au hasard dans un texte soient identiques. Comme pour
l'analyse des fréquences, cette indice varie selon la langue.
Ainsi, si l'indice de coïncidence d'un texte chiffré correspond à
celui d'une certaine langue, le texte est chiffré avec une
substitution monoalphabétique ou une transposition dans cette 
langue. Si par contre l'indice ne correspond à rien, on a affaire 
à une substitution polyalphabétique.

Nous noterons l'indice de coïncidence propre au langage $I_l$, et
l'indice de coïncidence correspondant à un texte où les lettres
seraient placées uniformément $I_m$ (c'est en fait l'indice
minimal). 
\\

Calculons cet indice :
\begin{itemize}
  \item La probabilité que deux lettres tirées dans un texte est
égal à la probabilité de prendre deux A, plus la probabilité de
prendre deux B, \dots L'indice de coïncidence du texte est donc la
somme de ces probabilités :
  \begin{center}
    \[ Ic = p(\mbox{prendre deux A}) + p(\mbox{prendre deux B}) +
\dots + p(\mbox{prendre deux Z}) \]
  \end{center}
  \item Comme nous avons $C^{n}_2$ façons de prendre deux lettres
d'un texte de $n$ lettres, et $C^{n_A}_2$ façons de prendre deux A
dans ce même texte, s'il contient $n_A$ A, la probabilité de prendre 
deux A est
  \begin{center}
    \[ p(\mbox{prendre deux A}) = \dfrac{C^{n_A}_2}{C^{n}_2} =
\dfrac{\dfrac{n_A!}{2 (n_A - 2)!}}{\dfrac{n!}{2 (n - 2)!}} = 
\dfrac{n_A (n_A - 1)}{n (n - 1)} \]
  \end{center}
  \item L'indice de coïncidence devient alors, pour $n_i$
représentant la $i$\ieme~ lettre de l'alphabet :
  \begin{center}
    \[ Ic = \frac{n_1 (n_1 - 1)}{n (n - 1)} + \frac{n_2 (n_2 - 1)}{n
(n - 1)} + \dots + \frac{n_{26} (n_{26} - 1)}{n (n - 1)} = \sum_{i = 1}^{i
+ 1} \frac{n_i (n_i - 1)}{n (n - 1)} \] 
  \end{center}
\end{itemize}

%TODO: grandes par
Dans le cas d'une substitution polyalphabétique, comme le chiffre
de Vigenère, on pourrait diviser le message chiffré en plusieurs
partie, chacune de ces parties étant l'équivalent d'une
substitution simple, sur laquelle on pourrait procéder à une
analyse fréquentielle. Le nombre de partie étant égal à la
longueur de la clé, le plus important est alors de trouver la
longueur de la clé.

Considérons un message de $n$ caractères, chiffré avec une clé de
longueur $k$. Si nous écrivons ce message dans un tableau de $k$
colonnes, nous nous retrouverons avec un tableau de $\frac{n}{k}$
lignes, où chaque colonne correspondrait à une substitution
simple.

Si maintenant, nous sélectionnons deux lettres dans ce tableau :
\begin{enumerate}
  \item Soit elles se trouvent dans deux colonnes différentes, 
nous avons
alors $C^k_2 ~ (\frac{n}{k})^2$ façons de les choisir. Le nombre
de paire de lettres identiques au sein de deux colonnes distinctes
est alors
\[ I_m ~ C^k_2 ~ \left(\dfrac{n}{k}\right)^2 = 
I_m ~ \dfrac{k (k - 1)}{2} ~
\left(\dfrac{n}{k}\right)^2 = I_m ~ \dfrac{n^2 ~ (k-1)}{2k} \]
  \item Soit elles se trouvent dans la même colonne, nous avons
alors $C^{\frac{n}{k}}_2 ~ k $ façons de les choisir. Le nombre de paire de lettres
identiques au sein d'une même colonne est
\[ I_l C^{\frac{n}{k}}_2 ~ k = I_l
\dfrac{n ~ (n-k)}{2k} \]
\end{enumerate}

On peut donc retrouver la valeur de l'indice de coïncidence du
texte :
\[Ic = \dfrac{I_m ~ \dfrac{n^2 ~ (k-1)}{2k} +
I_l \dfrac{n ~ (n-k)}{2k}}{C^n_2} =
\dfrac{I_m ~ n (k-1) + I_l (n-k)}{k ~
(n+1)} \]

De cette équation, on peut isoler $k$ :
\[k = \dfrac{n (I_l - I_m)}{n (Ic - I_m) + (I_l - Ic)} \]
Cette formule nous permet donc d'avoir une valeur assez proche de
la longueur de la clé.


\subsection{L'attaque par mot probable\label{sec:MotProbable}}
Quand on a la connaissance d'un mot qui pourrait être dans le
message chiffré, on a recours à cette attaque. On connaît alors le
mot en clair et en chiffré, on peut alors trouver des informations
pratique sur la clé, ou bien un bout ou l'entièreté de la clé.

Dans l'annexe \ref{Apx:FBI}, nous verrons une simple mise en pratique
de cette méthode, couplée avec l'analyse des fréquences.

\subsection{L'attaque par force brute}
L'attaque par force brute est une attaque qui théoriquement
fonctionne à tous les coups. Elle consiste à essayer toutes les
clés possibles, une à une, jusqu'à ce qu'on retrouve
la clé utilisée pour le chiffrement.

Bien évidemment, à cause des restrictions matérielles, cette
technique prends énormément de temps à trouver une solution. Ce
temps dépend de l'algorithme de chiffrement utilisé ainsi que de
la longueur de la clé. On comprends alors l'intérêt de choisir des
clés d'une certaine taille, afin de pouvoir résister à une attaque
par force brute (et cela ralentira aussi d'autres types
d'attaque).
\\

Il faut bien distinguer la longueur de la clé entre les
algorithmes de chiffrement symétriques et asymétriques, par
exemple, une clé de 1024 bits pour RSA (algorithme à clé
publique) correspond à une clé de 80 bits dans les algorithmes de
chiffrement symétriques. 

Dans le cas des chiffrements symétriques, pour une clé de $n$
bits, il existe alors $2^n$ clés possible, c'est donc autant
d'opérations qu'il faudra effectuer afin de trouver la bonne clé. 
\\

On considère un algorithme de chiffrement cassé lorsque le nombres
d'opérations à effectuer pour casser un message est plus
petit que le nombre d'opération qu'effectuerait une attaque par
force brute. Ainsi, dans le cas de la méthode de chiffrement
Triple DES, avec une clé de longueur de $2^{128}$ bits, une
attaque permet de casser un code en $2^{112}$ opérations (on dit
alors qu'il a 112 bits de sécurité). Ce nombre
reste très grand, mais l'algorithme est considéré comme cassé.
\\

Afin de pallier aux limites matérielles, il existe diverses
méthodes permettant d'avoir accès à une puissance de
calcul considérable :
\begin{itemize}
  \renewcommand{\makelabel}[1]{\sffamily\textbf{#1}}
  \item[Le calcul distribué] qui consiste comme son nom l'indique
à distribuer le calcul sur différentes machines, qui effectueront
les opérations en tâche de fond. C'est utilisé par exemple dans
les nombreux projets nommés « @home » comme « Folding@Home
»\footnote{Voir \url{http://folding.stanford.edu} pour plus
d'informations sur ce projet},
organisé par l'université de Stanford, et qui consiste à étudier
le repliement des protéines en fonction de diverses facteurs, dans
le but de fabriquer de nouveaux médicaments.
  \item[Les fermes de calcul (ou clustering)], qui, comme le
calcul distribué, permettent de rassembler les puissances de
calculs de plusieurs machines, sauf que dans ce cas ci, cette
puissance de calcul sera uniquement utilisée pour la tâche
désirée.
  \item[Les machines spécialisées], comme c'est le cas de
\emph{Deep Crack}, une machine mise au point en 1998 qui déchiffra
un message en 56 heures lors du DES Challenge.\\
\end{itemize}

On peut aussi citer une méthode utilisée par certains logiciel pour
cracker les sommes MD5, qui est d'utiliser la
puissance de calcul du processeur graphique (GPU) en plus de celle
du processeur (CPU). Ce genre de logiciel peut essayer plus de
$2^{28}$ empreintes MD5 par secondes sur un ordinateur 
personnel, et qui permet alors de retrouver le message clair en
moins d'une minute\footnote{Test disponible à l'adresse
\url{http://fz-corp.net/repository/MD5-GPU.html}}.

