\documentclass[10pt,a4paper]{memoir}

\usepackage{a4wide}
\usepackage[utf8]{inputenc}
\usepackage{a4}
%\usepackage{geometry}
\usepackage[francais]{babel}
\usepackage{hyperref}
\usepackage{listings}
%\usepackage{latexsym}
%\usepackage{graphicx}
%\usepackage{appendix}
\usepackage{fancybox}
\usepackage{url}
\usepackage{graphics}
\usepackage{graphicx}
\usepackage{harvard}
\urlstyle{sf}
\makeatletter\def\input@path{{contenu/}{src/}{eps/}}\makeatother


\newcommand{\crypt}[2]{\begin{center}\texttt{#1}\ \ $\Longrightarrow$\ \ \texttt{#2}\end{center}}
\newenvironment{fminipage}%
  {\begin{center}\begin{Sbox}\begin{minipage}{1\textwidth}}%
  {\end{minipage}\end{Sbox}\fbox{\TheSbox}\end{center}}
\newcommand{\definition}[1]{\begin{center}#1\end{center}}
\renewcommand{\footref}[1]{\footnote{voir chapitre \ref{#1}, page \pageref{#1}}}

%\setlength{\voffset}{-0.9in}
%\setlength{\hoffset}{-0.5in}
%\setlength{\textwidth}{6.5in}
%\setlength{\textheight}{10.5in}

\title{Travail de fin d'études \\
  \textbf{La cryptographie} \\ 
  Peut on réellement cacher des informations ?}
\author{Quentin Stiévenart\\
Athéné Royal de Waterloo}
\date{Année scolaire 2008 - 2009}


%\pagestyle{headings}

\begin{document}
% Titre, table des matières
\frontmatter
\maketitle \clearpage
\tableofcontents \clearpage

% Introduction
\chapter{Introduction}
\thispagestyle{empty}

\section{Présentation du travail}
La cryptographie fait aujourd'hui partie de la vie de tous les jours,
on l'utilise sans le savoir, principalement en naviguant sur internet
(via les connexions sécurisées, les envois de mails chiffrés\dots).

Mais qu'est-ce que la cryptographie ? Comment cela
fonctionne-t-il ? Son utilisation se limite-t-elle à l'informatique ?
Depuis quand existe la cryptographie ?

C'est à ces questions que nous allons essayer de répondre durant ce
travail, et plus principalement à celle-ci :

\begin{quote}
\emph{«~Peut-on réellement cacher des informations ?~»}
\end{quote}

Nous essayerons donc de savoir si l'on peut, et comment on peut 
envoyer sans risque une information, un message, des données\dots à
quelqu'un, sans qu'une personne extérieure puisse le lire.

Pour répondre à ces questions, nous allons tout d'abord parler de
l'utilisation  de la cryptographie
à travers l'histoire, pour présenter ensuite les
différentes techniques existantes. Nous parlerons
aussi du moyen de « contrer » la cryptographie, avec la
cryptanalyse. Nous verrons enfin ce qu'il en est de la
cryptographie de nos jours. 

% La cryptographie est, comme le dit si bien Lacroix dans le titre de
% son livre ``La cryptographie, ou l'art d'écrire en chiffres''
% \cite{ArtDecrireEnChiffres},
% un ensemble de méthodes mathématiques ou autres, par lesquelles on va ``crypter'' un message, de façon à le rendre illisible si on n'a pas connaissance de la méthode utilisée, ou de l'éventuelle clée utilisée. \\
% Ce travail aura pour but de savoir si la cryptographie est un moyen fiable pour cacher des données, via l'analyse de certains algorithmes utilisés de nos jours ou dans le passé, et plus précisément leur cryptanalyse (ce terme sera définit à la prochaine section du document) \\
% Nous analyserons donc, d'abord les méthodes de cryptographie utilisée dans l'histoire, suivant un ordre chronologique, ensuite nous passerons aux méthodes actuelles, où nous expliquerons entre autres l'utilité de la cryptographie de nos jours. \\

%En annexe, on pourra voir quelques implémentations d'algorithmes de chiffrement dans certains langages de programmation (Haskell, C, Common Lisp, Python)\footnote{Pour plus d'infos sur ces langages, visitez respectivement \url{http://www.haskell.org}, \url{http://fr.wikipedia.org/wiki/Langage_C}, \url{http://cliki.net} et \url{http://www.python.org}.} \\

\section{Mes motivations pour ce travail}
J'ai choisi de faire un travail de fin d'études sur la cryptographie
car j'ai toujours été intéressé par les mathématiques et surtout par
l'informatique (plus le fonctionnement de l'ordinateur et sa
programmation, que l'aspect « jeux et divertissements » de
l'informatique, ainsi que la sécurité informatique)~;
la cryptographie fait partie intégrante des
systèmes informatiques d'aujourd'hui et est basée sur des principes
mathématiques. Je compte d'ailleurs faire des études polytechniques
l'année prochaine. \\ L'idée de ce travail m'est venue petit à
petit. Après la lecture d'un article sur la cryptographie sur
internet, j'ai commencé à m'y intéresser un peu, mais sans plus.
 Ce travail me permet donc d'approfondir le sujet.

\section{À propos de ce document}
Pour des raisons personnelles et dans un esprit de liberté de
l'information, ce document est placé sous licence Creative Commons
BY-SA\footnote{\url{http://creativecommons.org/licenses/by-sa/2.0/fr/}}, ce qui
signifie que vous pouvez le modifier et le redistribuer librement à
condition de préciser le nom de l'auteur, et de garder cette même
licence. Toujours pour les mêmes raisons, ce document a été préparé à
l'aide d'outils entièrement libres.
%de l'éditeur de texte \texttt{GNU
%  Emacs} et du logiciel de composition typographique \LaTeX, et des
%logiciels \texttt{xfig}, \texttt{dia} et \texttt{inkscape} pour les
%dessins, sur un système d'exploitation basé sur \texttt{Linux}. Tous
%ces logiciels sont entièrement libres. De plus, les langages utilisés
%en annexe possèdent aussi chacun au moins une implémentation
%libre.
Pour plus d'informations sur les logiciels libres et la culture
libre, référez-vous à \url{http://www.gnu.org/philosophy/philosophy.fr.html}. \\
Ce document, ses sources (les fichiers \TeX, les graphiques, et les
codes sources des annexes), ainsi que des informations supplémentaires
sont disponibles sur internet via l'adresse
\url{http://acieroid.tuxfamily.org/crypto}. \\


\section{Présentation de la cryptologie}
Étymologiquement, la cryptologie signifie la «~science du secret~»~;
elle consiste à cacher une
information, un message ou de quelconques données via la
\emph{cryptographie}. Nous étudierons son histoire
dans le chapitre \ref{chap:Histoire}.

La cryptographie comprend des méthodes très variées (des
\emph{systèmes de chiffrement} pour rendre les messages chiffrés),
certaines méthodes étant plus efficaces que les autres.
Nous examinerons ces méthodes dans le chapitre
\ref{chap:Techniques}.

La cryptologie comprend aussi la \emph{cryptanalyse}, qui consiste à
retrouver le message d'origine à partir d'un message chiffré sans en
connaître la clé ou le secret utilisé pour chiffrer le message.
Nous verrons cet aspect dans le chapitre \ref{chap:Cryptanalyse}.

Au cours de ce travail, nous utiliserons de nombreux termes se référant à
la cryptologie. \\
Définissons-en quelques uns : 

%\newcommand{\glossaire}[2]{\nomenclature{\textbf{#1}}{#2}}
\begin{itemize}

\item \sffamily{\textbf{le chiffrement}} est la démarche effectuée afin de rendre
  le message clair illisible, chiffré. On utilise parfois le terme
  «~cryptage~», qui est un anglicisme, nous éviterons donc de
  l'utiliser~;

\item \sffamily{\textbf{le déchiffrement}} est la démarche inverse du chiffrement, qui retrouve
  le message clair à partir du message chiffré en ayant connaissance
  de la clé, du secret ou de l'algorithme utilisé (contrairement à la
  cryptanalyse). Le mot \sffamily{\textbf{décryptage}} peut aussi
être utilisé~;
\item \sffamily{\textbf{la cryptanalyse}} vise à retrouver le
message clair à partir du message chiffré sans avoir connaissance
de la clé ou du secret~;

\item \sffamily\textbf{{la stéganographie}} est une discipline semblable à la
  cryptographie mais qui consiste à cacher un message (dans une
  image, \dots) et non pas à la rendre initelligible. Nous ne
développerons donc pas ce sujet ici~;

\item \sffamily\textbf{{la clé de chiffrement}} est une donnée (mot, suite d'opération,
  nombre, \dots) utilisée pendant le chiffrement afin de rendre le
  déchiffrement plus difficile sans la connaissance de celle-ci. Il
  existe plusieurs types de clés : certaines qui doivent être gardées
  secrètes (clés privées) et d'autres qui peuvent être diffusées
  (clés publiques)~;

\item \sffamily\textbf{{les données claires}} sont les données dans leur forme initiale, non
  chiffrées. Ces données peuvent être du texte (un message) ou d'autres
  données informatiques (un fichier, \dots)~;

\item \sffamily\textbf{{les données chiffrées}} sont données qui
sont chiffrées via un certain
  algorithme de chiffrement~;

\item \sffamily\textbf{{un cryptogramme}} est un message chiffré
dans le but d'être déchiffré par des amateurs de cryptographie (on
en retrouve par exemple dans les journaux. Ils sont la plupart du
temps assez simples.

\end{itemize}



% Partie principale
\mainmatter

\chapter{Présentation de la cryptologie}
\section{Qu'est ce que la cryptologie}
La cryptologie est la « science du secret », et consiste a cacher une
information, un message, ou de quelconques données via la
\emph{cryptographie}. 

Elle comprends aussi la \emph{cryptanalyse}, qui consiste à retrouver
le message d'origine à partir d'un message chiffré, sans en connaître
la clé ou le secret utilisé pour chiffrer le message.
\section{Vocabulaire\label{s:Definitions}}
Avant toute chose, il est nécéssaire de définir quelques termes pour
les clarifier dans l'esprit de chacun, et pour éviter de les confondre
:
\begin{itemize}
\renewcommand{\makelabel}[1]{\sffamily\textbf{#1}}
\item[Le chiffrement] se réfère à la démarche effectuée afin de rendre
  le message clair illisible, chiffré. On utilise parfois le terme
  « cryptage », qui est un anglicisme, nous éviterons donc de
  l'utiliser ;

\item[Le déchiffrement, ou décryptage] est le sens inverse du
  chiffrement, qui retrouve le message clair à partir du message
  chiffré, en ayant connaissance de la clé, du secret ou de
  l'algorithme utilisé (contrairement à la cryptanalyse). Le terme
  « décryptage » est un terme français correct, contrairement au
  « cryptage » ;

\item[La stéganographie] est une discipline semblable à la
  cryptographie, mais qui consiste à « cacher » un message (dans une
  image, ...) et non pas à le rendre inintelligible. Ce sujet étant
  tout aussi vaste que la cryptographie, nous ne l'étudierons pas dans
  ce travail ;

\item[La clé de chiffrement] est une donnée (mot, suite d'opération,
  nombre, ...) utilisée pendant le chiffrement afin de rendre le
  déchiffrement plus difficile sans la connaissance de celle ci.
  Nous verrons qu'il existe deux types de clés : \emph{symétrique}
  (\emph{privées}) et \emph{asymétrique} (\emph{publiques}), et que
  l'utilisation de clé est crucial dans la cryptographie actuelle.
\end{itemize}

\section{Règles typographiques}
Tout au long de ce travail, nous allons utiliser certaines notations : 
\begin{itemize}
\item Une \textbf{clé de chiffrement} sera notée $k$ (pour «  Key  »,
  «  Clé  » en anglais), indicé si nous utilisons plusieurs clés ;
  \item Un \textbf{message en clair} sera noté $M$ (pour «  Message  ») ;
  \item Un \textbf{message chiffré} sera noté $C$ (pour «  Chiffré  ») ;
  \item Une \textbf{fonction de chiffrement} $e$ (pour « Encryption
    », «  Chiffrage  » en anglais) ou $e_k$, selon qu'elle utilise
    une clé pour le chiffrement ;
  \item Une \textbf{fonction de déchiffrement} $d$ (pour
    « Decryption », « Déchiffrement  » en anglais) ou $d_k$,
    similairement à la fonction de chiffrement.
\end{itemize}
L'opération de chiffrement se notera alors comme suit pour un
algorithme de chiffrement avec une clé.
\begin{center}
  \begin{math}
    e_k(M) = C
  \end{math}
\end{center}
Et l'opération de déchiffrement se notera ainsi :
\begin{center}
  \begin{math}
    d_k(C) = M
  \end{math}
\end{center}
Nous utiliserons aussi la notation : 
\crypt{Cryptographie}{Pelcgbtencuvr}
pour indiquer que le mot « Cryptographie » sera chiffré en
 « Pelcgbtencuvr ». 
  

\chapter{Les chiffres de substitution}
Cette méthode consiste à simplement remplacer une lettre, un ensemble
de lettres (un \emph{polygramme}), un mot, ...  par une autre lettre,
polygramme, ou bien par un nombre ou un signe.

On peut distinguer plusieurs sortes de substitutions : 
\begin{itemize}
  \renewcommand{\makelabel}[1]{\sffamily\textbf{#1}}
  \item[Les substitutions monoalphabétiques ou simples] : 
    chaque lettre est remplacée par une autre lettre tout au long du message~; 
  \item[Les substitutions polyalphabétiques] :
    c'est en fait une combinaison de substitutions simples~;
  \item[Les substitutions polygrammiques]: 
    à la place de substituer des lettres comme le fait la substitution
    monoalphabétique, on substitue des groupes de lettres~;
  \item[Les substitutions homophoniques] : 
    chaque lettre peut être remplacée par plusieurs valeurs, choisies
    aléatoirement.
\end{itemize}

\note{Dans ce chapitre, nous ferons abstraction des caractères
  spéciaux, comme les lettres avec accents, des signes de
  ponctuations, ou autre caractères. Nous remplacerons ceux-ci par une
  espace si besoin est.}

\section{Les substitutions monoalphabétiques}
Ce type de substitution consiste simplement à remplacer chaque lettre
de l'alphabet par une autre lettre.

\subsection{Par simple décalage}
On peut par exemple, comme le fait le chiffre de
César\footref{syst:chiffrecesar}, décaler les lettres de l'alphabet d'une
clé $k$. \\

 Si nous considérons l'alphabet comme un ensemble cyclique où 
chaque lettre correspondrait à un chiffre (une fois arrivé à $Z$ ($= 25$), on 
repart de $A$ ($= 0$) $\rightarrow 25 + 1 = 0$), on peut facilement définir 
ce système de chiffrement :  \\

\definition{$e_k(c) = (c + k)~ mod~ 26$\\
  où $c$ est le caractère à chiffrer.}

Nous utilisons l'opérateur \emph{modulo}, qui nous donne le reste de
la division entière de deux entiers ($5~ mod~ 2 = 1$), ce qui nous
permet de rester dans un ensemble cyclique. \\

\subsubsection{Exemple: le rot13\label{syst:rot13}}
Une application du chiffre de César est, par exemple, le \emph{ROT13},
qui est simplement un chiffre de César pour lequel la clé $k$ a une
valeur de $13$.\\

L'interêt du ROT13 réside dans le fait qu'il suffit d'appliquer deux
fois la méthode de chiffrement pour retrouver le texte en clair
($e_{13}(e_{13}(M)) = M$), ainsi, il n'y a qu'une et une seule fonction utilisée
pour le chiffrement et le déchiffrement (donc $e_{13}(x) =
d_{13}(x)$). \\

De ce fait, ROT13 n'est pas du tout pratique pour cacher des
informations, mais est parfois utilisé sur internet, dans les forums,
afin d'empêcher la lecture involontaire d'un texte (qui pourrait
dévoiler une information sur un film par exemple, et gâcherait le
plaisir au gens n'ayant pas vu le film, c'est ce qu'on appelle un 
\emph{spoiler}) 

\subsection{Par remplacement}
Outre le fait de décaler les lettres, on peut aussi remplacer chaque
lettre de l'alphabet par une autre lettre. Pour chiffrer, on peut alors
s'aider d'une table de substitution, comme sur la figure
\ref{fig:substitutionsimple}.

 \begin{figure}[h]
   \begin{center}
    \begin{tabular}{|c|c|c|c|c|c|c|c|c|c|c|c|c|c|c|c|c|c|c|c}
      \hline
      A & B & C & D & E & F & G & H & I & J & K & L & M & N & O & P &
      Q & R & S & T \\
      \hline
      A & Z & E & R & T & Y & U & I & O & P & Q & S & D & F & G & H &
      J & K & L & M \\
      \hline
    \end{tabular}
  \end{center}
  \begin{flushright}
    \begin{tabular}{c|c|c|c|c|c|}
      \hline
      U & V & W & X & Y & Z \\
      \hline
      W & X & C & V & B & N \\
      \hline
    \end{tabular}
  \end{flushright}
  \caption{Exemple simple de table de substitution}
  \label{fig:substitutionsimple}
\end{figure}

Dans cette table de substitution, on peut placer les lettre
aléatoirement, selon une règle, ou en utilisant une clé (qu'on notera
en début de table, et nous recopierons les caractères non utilisés par
après, voir la figure \ref{fig:substitutioncle}).

\begin{figure}[h]
  \begin{center}
    \begin{tabular}{|c|c|c|c|c|c|c|c|c|c|c|c|c|c|c|c|c|c|c|c}
      \hline
      A & B & C & D & E & F & G & H & I & J & K & L & M & N & O & P &
      Q & R & S & T \\
      \hline
      C & R & Y & P & T & O & A & B & D & E & F & G & H & I & J & K &
      L & M & N  & Q  \\
      \hline
    \end{tabular}
  \end{center}
  \begin{flushright}
    \begin{tabular}{c|c|c|c|c|c|}
      \hline
      U & V & W & X & Y & Z  \\
      \hline
      S & U & V & W & X & Z \\
      \hline
    \end{tabular}
  \end{flushright}
  \caption{Exemple de table de substitution avec comme clé ``crypto''}
  \label{fig:substitutioncle}
\end{figure}

Il existe de nombreuses méthodes dans le même genre pour remplacer
simplement une lettre par une autre, voyons en une de plus près :
\emph{le carré de Polybe.}\\

\subsection{Exemple: le carré de Polybe\label{syst:carrepolybe}}
Polybe est un historien grec (210 av. J.-C., 126 av. J.-C.), qui
explique vers 150 av. J.-C. une méthode de chiffrement par
substitution assez simple et intéressante.\\

Cette méthode consiste à placer les lettres de l'alphabet dans un
carré de 25 cases (il faut donc retirer une lettre, le W pour le
français, qui sera remplacé par V dans le message). La lettre chiffrée
sera remplacée par un nombre formé par le numéro de la colonne et de
la ligne de la lettre.\\

Nous pouvons représenter ce carré par une matrice de taille $5\times5$
(voir la figure \ref{fig:polybe}). Nous pouvons, analogiquement à la
figure \ref{fig:substitutioncle}, utiliser une clé pour le carré de
Polybe. \\

Le carré de Polybe est assez intéressant car il convertit les lettres
en chiffres et réduit le nombre de symbole utilisés dans le message
chiffrés (9 chiffres plutôt que 26 lettres). \\

\begin{figure}[h]
  $
  \left(
    \begin{array}{ccccc}
      A & B & C & D & E \\
      F & G & H & I & J \\
      K & L & M & N & O \\
      P & Q & R & S & T \\
      U & V & X & Y & Z
    \end{array}
  \right)
  $
  \hfill
  $
  \left(
    \begin{array}{ccccc}
      C & R & Y & P & T \\
      O & A & B & D & E \\
      F & G & H & I & J \\
      K & L & M & N & Q \\
      S & U & V & X & Z
    \end{array}
  \right)
  $
  \hfill
  $
  \left(
    \begin{array}{ccccc}
      a_{11} & a_{12} & a_{13} & a_{14} & a_{15}  \\
      a_{21} & a_{22} & a_{23} & a_{24} & a_{25}  \\
      a_{31} & a_{32} & a_{33} & a_{34} & a_{35}  \\
      a_{41} & a_{42} & a_{43} & a_{44} & a_{45}  \\
      a_{51} & a_{52} & a_{53} & a_{54} & a_{55}
    \end{array}
  \right)
  $
  \caption{Le carré de Polybe sous forme de matrice, la seconde
    matrice ayant comme clé « crypto », et la troisième représentant
    les nombres par lesquels seront remplacés les caractères dans le
    message chiffré.}
  \label{fig:polybe}
\end{figure}

Polybe imaginait un moyen de communiquer les messages via des torches
: le nombre de torches placée à gauche correspondrait au numéro de la
ligne, et le nombre de torches à droit au numéro de la colonne. \\

Le carré de Polybe était utilisé par les nihilistes russes entre le
XIX\ieme~ et le XX\ieme~ siècle (des « terroristes » dont le but était de
% TODO: eme
tuer le tsar pour reconstruire une société sur de nouvelles
bases). Lorsqu'ils étaient attrapés et enfermés en prison, ils
communiquaient via le carré de Polybe, en donnant des coups sur les
murs. \\ %TODO: mieux expliquer

\begin{figure}[h]
  \begin{center}
    \includegraphics[scale=1.5]{eps/nihilistes}
  \end{center}
  \caption{Version du carré de Polybe utilisée par les nihilistes
    russes}
  \label{fig:nihilistes}
\end{figure}

\section{Les substitutions polyalphabétiques\label{substitutionpolyalphabetique}}
Les substitutions polyalphabétiques sont une combinaison de plusieurs
tables de substitutions simples, où l'on change de table à chaque
lettre, ce qui rend le message chiffré beacoup plus dur à casser sans
le code (nous verrons ceci en détail dans le chapitre
\ref{cryptanalyse}, page \pageref{cryptanalyse} sur la cryptanalyse)

Une des substitutions polyalphabétiques les plus connues est le
\emph{Chiffre de Vigenère}.

\subsection{Exemple : Le chiffre de Vigenère\label{syst:chiffrevigenere}}
Au XIX\ieme~ siècle, Vigenère « invente » une nouvelle méthode de
chiffrement en s'inspirant fortement des travaux de l'abbé allemand
Jean Trithème du XVI\ieme~ siècle. Blaise de Vigenère a néanmoins
modifié légèrement la méthode de Trithème en rendant l'utilisation de
clé de chiffrement possible. \\

La technique de Jean Trithème est d'utiliser ce qu'il décrit comme une
\emph{tabula recta}
(voir figure \ref{fig:tabularecta}) qui est composée de 24 rangées de 24
lettres chacune. On retire donc deux lettres, le j et le v, qui seront
respectivement remplacées par le i et le w. \\
Chaque rangée correspond à un chiffre de César\footref{syst:chiffrecesar},
avec à chaque fois un décalage augmenté d'une unité. \\

Par exemple, pour chiffrer le mot « cryptographie » : 
\begin{itemize}
  \item la lettre C est chiffrée sur la première colonne, et reste
    donc C~;
  \item la lettre R est chiffrée sur la 17\ieme~ colonne, et devient
    S. Pour cela, trouvez la colonne de R dans la première rangée, et
    descendez d'une colonne~;
  \item La lettre Y, dans l'avant dernière colonne devient A (même
    colonne, deux rangées plus bas)~;
  \item Et ainsi de suite ~\dots~;
  \item « cryptographie » sera donc chiffré en « csasztnwiwsur ».
\end{itemize}

\begin{figure}[h]
  \begin{center}
%     \begin{tabular}{c@{}c@{}c@{}c@{}c@{}c@{}c@{}c@{}c@{}c@{}c@{}c@{}c@{}c@{}c@{}c@{}c@{}c@{}c@{}c@{}c@{}c@{}c@{}c}
%       A & B & C & D & E & F & G & H & I & K & L & M & N & O & P &
%       Q & R & S & T & U & W & X & Y & Z  \\

%       B & C & D & E & F & G & H & I & K & L & M & N & O & P &
%       Q & R & S & T & U & W & X & Y & Z & A \\

%       C & D & E & F & G & H & I & K & L & M & N & O & P &
%       Q & R & S & T & U & W & X & Y & Z & A & B \\

%       D & E & F & G & H & I & K & L & M & N & O & P &
%       Q & R & S & T & U & W & X & Y & Z & A & B & C \\
      
%       \dots & \\
      
%       Z & A & B & C & D & E & F & G & H & I & K & L & M & N & O & P &
%       Q & R & S & T & U & W & X & Y \\

%     \end{tabular}
%    \hfill
    \includegraphics[scale=0.3]{eps/tabularecta}
  \end{center}
  \caption{La \emph{tabula recta} de Jean Trithème}
  \label{fig:tabularecta}
\end{figure}

Vigenère reprends donc cette méthode pour légèrement la modifier (y
ajouter l'utilisation d'une clé en fait), et la publie dans son livre
\emph{Traicté des chiffres, ou secretes
  manieres d'escrire} en 1586. \\

Comme pour la méthode de Trithème, on utilise une table composée des
26 chiffres de César possibles, comme sur la figure
\ref{fig:vigeneretableau} (Trithème en décrivait 24 dans son livre,
mais on pourrait très bien en utiliser 26). \\

Expliquons cette méthode au travers d'un exemple : chiffrons « chiffre
de vigenere » avec comme mot cle « crypto ».
\begin{itemize}
  \item Tout d'abord, chaque lettre du message est associée à une
    lettre de la clé, de la façon suivante : \\
    \begin{tabular}{c@{}c@{}c@{}c@{}c@{}c@{}cc@{}cc@{}c@{}c@{}c@{}c@{}c@{}c@{}c}
      c & h & i & f & f & r & e &
      d & e &
      v & i & g & e & n & e & r & e \\

      c & r & y & p & t & o & c & 
      r & y & 
      p & t & o & c & r & y & p & t \\
    \end{tabular}\\
    La clé est donc répétée successivement~;
  \item Ensuite, chaque lettre du message clair est décalée du nombre
    de rang correspondant à la lettre de la clé qui lui est
    associée. Ainsi, pour la première lettre, C, on la décale de 2
    rangs (la lettre C lui étant associée, et la place de cette lettre
    dans l'alphabet en partant de 0 est 2). Pour s'aider, on peut
    utiliser la table du chiffre de Vigenère (voir la figure
    \ref{fig:vigeneretableau}). \\
    Ainsi, C est chiffré en E.
  \item Et on continue ainsi de suite, H est chiffré en Y, \dots
  \item « chiffre de vigenere » sera donc chiffré en « eyguyfg uc
    kbugecgx »
\end{itemize}

\begin{figure}[h]
  \begin{center}
    \begin{tabular}{c|c@{}c@{}c@{}c@{}c@{}c@{}c@{}c@{}c@{}c@{}c@{}c@{}c@{}c@{}c@{}c@{}c@{}c@{}c@{}c@{}c@{}c@{}c@{}c@{}c@{}c}
        & A & B & C & D & E & F & G & H & I & J & K & L & M & N & O & P & Q & R & S & T & U & V & W & X & Y & Z \\
      \hline
      A & A & B & C & D & E & F & G & H & I & J & K & L & M & N & O & P & Q & R & S & T & U & V & W & X & Y & Z \\
      B & B & C & D & E & F & G & H & I & J & K & L & M & N & O & P & Q & R & S & T & U & V & W & X & Y & Z & A \\
      C & C & D & E & F & G & H & I & J & K & L & M & N & O & P & Q & R & S & T & U & V & W & X & Y & Z & A & B \\
      D & D & E & F & G & H & I & J & K & L & M & N & O & P & Q & R & S & T & U & V & W & X & Y & Z & A & B & C \\
      E & E & F & G & H & I & J & K & L & M & N & O & P & Q & R & S & T & U & V & W & X & Y & Z & A & B & C & D \\
      G & G & H & I & J & K & L & M & N & O & P & Q & R & S & T & U & V & W & X & Y & Z & A & B & C & D & E & F \\
      H & H & I & J & K & L & M & N & O & P & Q & R & S & T & U & V & W & X & Y & Z & A & B & C & D & E & F & G \\
      I & I & J & K & L & M & N & O & P & Q & R & S & T & U & V & W & X & Y & Z & A & B & C & D & E & F & G & H \\
      J & J & K & L & M & N & O & P & Q & R & S & T & U & V & W & X & Y & Z & A & B & C & D & E & F & G & H & I \\
      K & K & L & M & N & O & P & Q & R & S & T & U & V & W & X & Y & Z & A & B & C & D & E & F & G & H & I & J \\
      L & L & M & N & O & P & Q & R & S & T & U & V & W & X & Y & Z & A & B & C & D & E & F & G & H & I & J & K \\
      M & M & N & O & P & Q & R & S & T & U & V & W & X & Y & Z & A & B & C & D & E & F & G & H & I & J & K & L \\
      N & N & O & P & Q & R & S & T & U & V & W & X & Y & Z & A & B & C & D & E & F & G & H & I & J & K & L & M \\
      O & O & P & Q & R & S & T & U & V & W & X & Y & Z & A & B & C & D & E & F & G & H & I & J & K & L & M & N \\
      P & P & Q & R & S & T & U & V & W & X & Y & Z & A & B & C & D & E & F & G & H & I & J & K & L & M & N & O \\
      Q & Q & R & S & T & U & V & W & X & Y & Z & A & B & C & D & E & F & G & H & I & J & K & L & M & N & O & P \\
      R & R & S & T & U & V & W & X & Y & Z & A & B & C & D & E & F & G & H & I & J & K & L & M & N & O & P & Q \\
      S & S & T & U & V & W & X & Y & Z & A & B & C & D & E & F & G & H & I & J & K & L & M & N & O & P & Q & R \\
      T & T & U & V & W & X & Y & Z & A & B & C & D & E & F & G & H & I & J & K & L & M & N & O & P & Q & R & S \\
      U & U & V & W & X & Y & Z & A & B & C & D & E & F & G & H & I & J & K & L & M & N & O & P & Q & R & S & T \\
      V & V & W & X & Y & Z & A & B & C & D & E & F & G & H & I & J & K & L & M & N & O & P & Q & R & S & T & U \\
      W & W & X & Y & Z & A & B & C & D & E & F & G & H & I & J & K & L & M & N & O & P & Q & R & S & T & U & V \\
      X & X & Y & Z & A & B & C & D & E & F & G & H & I & J & K & L & M & N & O & P & Q & R & S & T & U & V & W \\
      Y & Y & Z & A & B & C & D & E & F & G & H & I & J & K & L & M & N & O & P & Q & R & S & T & U & V & W & X \\
      Z & Z & A & B & C & D & E & F & G & H & I & J & K & L & M & N & O & P & Q & R & S & T & U & V & W & X & Y \\
    \end{tabular}
  \end{center}
  \caption{Le tableau utilisé pour le chiffre de Vigenère}
  \label{fig:vigeneretableau}
\end{figure}

Le chiffre de Vigenère consiste donc à « additionner » la lettre du
message clair à la lettre de la clé : \\
\begin{center}
$e_k(C) = C + K$
\end{center}

Il existe alors de nombreuses variantes au chiffre de Vigenère qui
consistent à effectuer d'autres opérations simples avec le caractère
clair et le caractère de la clé (facilement faisable à la main) ,
comme on peut en voir quelques unes dans la table \ref{tab:variantesvigenere}. \\
\begin{table}[h]
  \caption{Quelques variantes du chiffre de Vigenère}
  \label{tab:variantesvigenere}
  \begin{center}
    \begin{tabular}{|l|c|}
      \hline
      \textbf{Nom du chiffre} & \textbf{Opérations} \\
      \hline
      Chiffre de Vigenère & $e_k(C) = C + K$ \\ 
      \hline
      Chiffre de Beaufort & $e_k(C) = K - C$ \\
      \hline
      Chiffre de Beaufort, variante allemande & $e_k(C) = C - K$ \\
      \hline
    \end{tabular}
  \end{center}
\end{table}
%TODO: expliquer les méthodes à la main ? http://www.apprendre-en-ligne.net/crypto/menu/index.html

\section{Substitution homophonique}
% Chaque lettre peut être remplacée par plusieurs lettres/chiffres
\section{Substitution polygrammique}
% Substitution sur plusieurs caractère.
% \subsection{Les substitutions alphabétiques}
% Les substitution alphabétiques consistent simplement à remplacer une lettre par une autre, on retrouve deux types de substitutions alphabétiques : 
% \subsubsection{Les substitution mono-alphabétique}
% On remplace chaque lettre de l'alphabet par une autre lettre, ou une séquence de chiffre (comme le carré de Polybe, voir \ref{CarrePolybe}, page \pageref{CarrePolybe}), qui restera toujours la même tout au long du cryptage. On peut utiliser une table de substitution pour faciliter le chiffrement. La première ligne correspond au caractère en clair, à chiffrer, et la seconde ligne au caractère chiffré.

% \begin{figure}[h]
%     \begin{center}
%     \begin{tabular}{|c|c|c|c|c|c|c|c|c|c|c|c|c|c|c|c|c|c|c|c|c|c|c|c|c|c|}
%       \hline
%       A & B & C & D & E & F & G & H & I & J & K & L & M & N & O & P & Q & R & S & T
%       & U & V & W & X & Y & Z \\
%       \hline
%       A & Z & E & R & T & Y & U & I & O & P & Q & S & D & F & G & H & J & K & L & M
%       & W & X & C & V & B & N \\
%       \hline
%     \end{tabular}
%   \end{center}
%   \caption{Exemple simple de table de substitution}
%   \label{TableSubstitutionSimple}
% \end{figure}
% De cette façon, suivant la table présente dans la figure \ref{TableSubstitutionSimple}, si on crypte ``bonjour'' : 
% \crypt{bonjour}{zgfpgwk}

% On peut décaler simplement des lettres de l'alphabet, comme c'est le cas pour le chiffre de César (voir \ref{ChiffreCesar}, page \pageref{ChiffreCesar}),
%  on peut utiliser une clé (voir figure \ref{TableSubstitutionCle}), où remplacer simplement les lettres par d'autres lettres arbitrairement (voir figure \ref{TableSubstitutionSimple}).
% \begin{figure}[h]
%   \begin{center}
%     \begin{tabular}{|c|c|c|c|c|c|c|c|c|c|c|c|c|c|c|c|c|c|c|c|c|c|c|c|c|c|}
%       \hline
%       A & B & C & D & E & F & G & H & I & J & K & L & M & N & O & P & Q & R & S & T
%       & U & V & W & X & Y & Z \\
%       \hline
%       C & R & Y & P & T & O & A & B & D & E & F & G & H & I & J & K & L & M & N  & Q 
%       & S & U & V & W & X & Z \\
%       \hline
%     \end{tabular}
%   \end{center}
%   \caption{Exemple de table de substitution avec comme clé ``crypto''}
%   \label{TableSubstitutionCle}
% \end{figure}
% Nous verrons dans le chapitre sur la cryptanalyse (page \pageref{Cryptanalyse})qu'il est facile de casser ces substitution via l'analyse des fréquences


%\part{ Annexes}
\appendix

% Fin du travail : La bibliographie, le glossaire, à propos
\backmatter
\nocite{*}
\bibliographystyle{agsm}
\bibliography{bibliographie/bibliographie}

\end{document}
